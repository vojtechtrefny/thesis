\documentclass{beamer}

\mode<presentation> {
%  \usetheme{Warsaw}
	\usetheme{Boadilla}
	\setbeamercovered{transparent}
}

%\usepackage{ucs}
\usepackage[utf8]{inputenc}
\usepackage[czech]{babel}
%\usepackage{palatino}
\usepackage{graphicx}
\usepackage{listings}

\title[Bitlocker šifrování v Linuxovém prostředí]{Bitlocker šifrování v Linuxovém prostředí\\\small{Diplomová práce -- kontrolní den č. 1}}
\author{Vojtěch Trefný}
\institute[FAI UTB]{Fakulta aplikované informatiky UTB}
\date{1.~3.~2019}

\begin{document}

\begin{frame}
	\titlepage
\end{frame}

\begin{frame}
	\frametitle{Osnova}
	\tableofcontents
\end{frame}

%%%%%%%%%%%%%%%%%%%%%%%%%%%%%%%%%%%%%%%%%%%%%%%%%%%%%%%%%%%%%%%%%%

\section{Zadání}

\begin{frame}
	\frametitle{Zadání}
	\begin{block}{}
		\begin{itemize}
			\item \textbf{Vedoucí:} Ing. Michal Bližňák Ph.D.
			\item \textbf{Konzultant:} Ing. Milan Brož (Red Hat Czech/CRoCS FI MUNI)
		\end{itemize}
	\end{block}

	\begin{block}{}
		\begin{itemize}
			\item Seznamte se s nástrojem Windows Bitlocker pro šifrování disků.
			\item Popište podporované šifrovací módy a možnosti správy klíčů.
			\item Analyzujte použitá kryptografická primitiva a jejich atributy.
			\item Seznamte se s nástrojem a knihovnou libbde a možnostmi přístupu k Bitlocker obrazu disku v prostředí OS Linux.
			\item Navrhněte a podle možností implementujte nutná rozšíření Linuxových nástrojů pro jednoduchý přístup k obsahu Bitlocker disku.
		\end{itemize}
	\end{block}

\end{frame}

%%%%%%%%%%%%%%%%%%%%%%%%%%%%%%%%%%%%%%%%%%%%%%%%%%%%%%%%%%%%%%%%%%

\section{Rešerše}

\begin{frame}
	\frametitle{Dostupné zdroje informací}
	\begin{block}{}
		\begin{itemize}
			\item Původní implementace pro Windows Vista je popsána Nielsem Fergusonem v článku \emph{AES-CBC + Elephant diffuser A Disk Encryption Algorithm for Windows}
			\item Novější varianty částečně popisuje Dan Rosendorf v článku \emph{Bitlocker: A little about the internals and what changed in Windows 8}.
			\item Velmi dobrou specifikaci formátu BitLocker obsahuje také dokumentace ke knihovně \texttt{libbde} od Joachima Metze.
			\item Existují i další zdroje, které se většinou věnují prvním verzím BitLockeru v době jeho vzniku v roce 2006.
		\end{itemize}
	\end{block}

\end{frame}

\begin{frame}
	\frametitle{Podpora použitých kryptografických funkcí}

	\begin{block}{Userspace}
		\begin{itemize}
			\item Používané kryptografické algoritmy:
			\begin{itemize}
				\item AES-CCM
				\item SHA256
			\end{itemize}
			\item Plně podporované ve standardních kryptografických knihovnách (\texttt{libopenssl}, \texttt{libgcrypt}).
		\end{itemize}
	\end{block}

\vspace{0.5cm}

	\begin{block}{Kernel}
		\begin{itemize}
			\item Používané kryptografické algoritmy:
			\begin{itemize}
				\item AES-CBC 128/256bit (Windows Vista)
				\item AES-CBC 128/256bit + Elephant Diffuser (Windows Vista)
				\item AES-XTS 128/256bit (Windows 7+)
			\end{itemize}
			\item V kernel crypto API podporavané kromě Elephant.
		\end{itemize}
	\end{block}

\end{frame}

%%%%%%%%%%%%%%%%%%%%%%%%%%%%%%%%%%%%%%%%%%%%%%%%%%%%%%%%%%%%%%%%%%

\section{Implementace prototypu}

\begin{frame}
	\frametitle{Prototyp pro práci s BitLockerem v Linuxu}

	\begin{block}{}
		\begin{itemize}
			\item Jednoduchý \uv{proof-of-concept} napsaný v Pythonu s použitím knihovny \texttt{pycryptoraphy}.
			\item Pouze základní podpora pro data šifrovaná pomocí AES-XTS (BitLocker varianta ve Windows 7+).
		\end{itemize}
	\end{block}

\vspace{0.5cm}

	\begin{block}{}
		\begin{itemize}
			\item V současné době zvládá:
			\begin{itemize}
				\item Odvodit dešifrovací klíč z hesla nebo záložního (recovery) hesla.
				\item Dešifrovat klíče uložené v BitLocker hlavičce (\texttt{VMK} a \texttt{FVEK}).
				\item Dešifrovat první sektor disku (NTFS hlavička) pomocí \texttt{FVEK}.
			\end{itemize}
		\end{itemize}
	\end{block}

\end{frame}

\begin{frame}[fragile]
	\frametitle{Ukázka -- BitLocker hlavička}

	\begin{lstlisting}
Encryption:	AES-XTS 128-bit encryption
Identifier:	1f8bf933-8323-4c97-8a89-a67625ac8f40
Creation time:	2019-02-03 09:10:22.265406
Description:	DESKTOP-NPM7RCA G: 2/3/2019

VMK
	Identifier:	f0f61678-fb6f-4ab1-934a-...
	Type:		VMK protected with password
	Salt:		03 d1 b4 23 6b f4 5b df ...
	AES-CCM encrypted key
		Nonce:	2019-02-03 09:10:36.052000
		Count:	3
		Key:	0d a8 61 01 ...
	\end{lstlisting}

\end{frame}

\begin{frame}[fragile]
	\frametitle{Ukázka -- BitLocker první sektor}

	\begin{lstlisting}
00000000: eb 52 90 ... 08 00 00	|.R.NTFS     .....|
00000010: 00 00 00 ... 28 03 00	|........ ?....(..|
00000020: 00 00 00 ... 00 00 00	|........ ........|
00000030: 55 21 00 ... 00 00 00	|U!...... ........|
00000040: f6 00 00 ... 3d 84 a4	|........ RS=.}=..|
...
00000180: b4 0e bb ... 20 64 69	|........ ....A di|
00000190: 73 6b 20 ... 20 6f 63	|sk read  error oc|
000001a0: 63 75 72 ... 4d 47 52	|curred.. .BOOTMGR|
000001b0: 20 69 73 ... 64 00 00	| is comp ressed..|
000001c0: 0a 50 72 ... 6c 74 2b	|.Press C trl+Alt+|
000001d0: 44 65 6c ... 74 0d 0a	|Del to r estart..|
000001e0: 00 00 00 ... 00 00 00	|........ ........|

	\end{lstlisting}

\end{frame}

%%%%%%%%%%%%%%%%%%%%%%%%%%%%%%%%%%%%%%%%%%%%%%%%%%%%%%%%%%%%%%%%%%

\section{Další kroky}

\begin{frame}
  \frametitle{Další kroky}
	\begin{block}{}
		\begin{itemize}
					\item Rozšíření současného prototypu o podporu pro čtení celého šifrovaného disku.
					\item Testování s použitím standardních nástrojů pro tvorbu blokových zařízení v Linuxu (\texttt{device-mapper}/\texttt{dmsetup}).
					\item Případné rozšíření \texttt{dm-crypt} modulu o chybějící funkcionalitu (pravděpodobně podpora odvození IV).
					\item Implementace prototypu jako knihovny v jazyce C tak, aby jej šlo použít v existujících nástrojích/knihovnách jako \texttt{cryptsetup} a/nebo \texttt{UDisks}.
					\item Podpora ostatních (starších a méně obvyklých) variant BitLockeru.
		\end{itemize}
	\end{block}

\end{frame}

%%%%%%%%%%%%%%%%%%%%%%%%%%%%%%%%%%%%%%%%%%%%%%%%%%%%%%%%%%%%%%%%%%

\section{Závěr}

\begin{frame}
	\frametitle{Závěr}

	\begin{center}
	Děkuji vám za pozornost.
	\end{center}

\vspace{0.5cm}

	\begin{center}
	Prostor pro vaše dotazy.
	\end{center}
\end{frame}

\end{document}
