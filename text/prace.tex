% ============================================================================ %
%
%           Šablona bakalářské/diplomové práce
%
% Autor:    Ing. Jozef Říha (4. květen 2006)
%           (některé komentáře převzaty z dokumentu Ivana Pomykacze)
%
% Verze:    2017-01-19, Ing. Pavel Tomášek (tomasek@fai.utb.cz)
%
% Kódování: UTF-8 (žluťoučký kůň úpěl ďábelšké ódy)
%
% Sazba:    pdflatex prace.tex && pdflatex prace.tex
%           (nutné dvakrát pro korektní vložení citací a jiných referencí),
%           v případě umístění literatury do externího bib souboru je třeba volat
%           pdflatex statement.tex && bibtex statement.tex && pdflatex statement.tex && pdflatex statement.tex
%
% Tip:      Ve správně vysázeném českém textu by na konci řádku neměla zůstant
%           samotná jednopísmenná předložka. Na takové místo se vkládá
%           nezalomitelná mezera pomocí symbolu ~. Existuje program, který umí
%           zpracovat celý TeX dokument najednou podle českých konvencí:
%           http://petr.olsak.net/ftp/olsak/vlna/
%
%           Pozor! Vzhledem k požadovanému standardu PDF/A nesmí obrázky obsahovat 
%           alfa kanál (průhlednost).
%
% ============================================================================ %


\documentclass[a4paper,12pt]{article}

% Definice vzhledu a nastavení se načítá z následujícího souboru (netřeba editovat)
% ============================================================================ %
% Tento dokument není zpravidla třeba editovat,
% obsahuje nastavení balíčků, vzhledu, stylů.
%
% Kódování: UTF-8 (žluťoučký kůň úpěl ďábelšké ódy)
% ============================================================================ %


% ============================================================================ %
% BALÍČKY

%\usepackage[czech,english]{babel} % volba při kompilaci latexem (vyžaduje texlive-lang), zakomentovano, nastavovanu prikazem \nastavjazyk
\usepackage[T1]{fontenc}% definice vnitřního kódování
\usepackage[utf8x]{inputenc} % slouží pro definici kódování (při problémech zkusit zaměnit utf8x za utf8)
\usepackage{color}      % umožňuje použití barev
\usepackage{graphicx}	% rozšíření práce s grafikou
\usepackage{amsmath}	% balíček pro pokročilejší matematiku
\usepackage{fancyhdr}	% detailnější nastavení záhlaví a zápatí
\usepackage{tocloft}	% umožňuje pohodlné nastavení vzhledu obsahu, seznamu tabulek či obrázků
\usepackage{textcase}	% změna VeLiKoStI PíSmA
\usepackage{ifthen} 	% balíček umožňující skladby if, then -- využijeme při definici nadpisů
\usepackage{setspace}	% balíček umožňující nastavit řádkování na 1, 1.5, 2
\usepackage{ccaption}	% vylepšení práce s popisky obrázků či tabulek
\usepackage{sectsty}	% pro nastavení vzhledu nadpisů
\usepackage[srcstyle=leftnumhang,linenumbersep={\ }]{examplep} % pokročilejší sazba programového kódu
\usepackage{url}        % balíček pro vysázení internetové adresy stylem verbatim s vylepšeným řádkovým zlomem
\usepackage{afterpage}
%\usepackage{layout}	% zobrazí nastavení tiskového zrcadla (příkaz \layout)
%\usepackage{times}		% balíček pro použití fontu times
%\usepackage{verbatim}	% vysází text bez formátování, tak jak je zapsán v souboru
%\usepackage{indentfirst} % definuje odsazení prvního řádku odstavce
%\usepackage{makeidx}	% vytvoří rejstřík
\usepackage[pdftex,pdfa,hidelinks,breaklinks]{hyperref}	% vytváří křížové odkazy
%\usepackage{multicol}	% vícesloupcová sazba
%\usepackage{flafter}	% zajistí, aby se plovoucí objekty objevovali až za jejich umístěním v textu
\usepackage{chngcntr}   % Umožňuje změnu nastavení číslování obrázků, tabulek i rovnic
\usepackage{etoolbox}   % Tool-box for LaTeX programmers
\usepackage[labelsep=space,tableposition=bottom,justification=centering]{caption} % Přenastavení popisků u figur a tabulek

% ---------------------------------------------------------------------------- %

\usepackage{xmpincl}
\usepackage{hyperxmp}

% \convertDate converts D:20080419103507+02'00' to 2008-04-19T10:35:07+02:00
\def\convertDate{%
    \getYear
}
{\catcode`\D=12
 \gdef\getYear D:#1#2#3#4{\edef\xYear{#1#2#3#4}\getMonth}
}
\def\getMonth#1#2{\edef\xMonth{#1#2}\getDay}
\def\getDay#1#2{\edef\xDay{#1#2}\getHour}
\def\getHour#1#2{\edef\xHour{#1#2}\getMin}
\def\getMin#1#2{\edef\xMin{#1#2}\getSec}
\def\getSec#1#2{\edef\xSec{#1#2}\getTZh}
\def\getTZh +#1#2{\edef\xTZh{#1#2}\getTZm}
\def\getTZm '#1#2'{%
    \edef\xTZm{#1#2}%
    \edef\convDate{\xYear-\xMonth-\xDay T\xHour:\xMin:\xSec+\xTZh:\xTZm}%
}
\expandafter\convertDate\pdfcreationdate

\pdfminorversion 4

\immediate\pdfobj stream attr{/N 3}  file{graphics/sRGBIEC1966-2.1.icm}
\pdfcatalog{%
  /OutputIntents [ <<
  /Type /OutputIntent
  /S/GTS_PDFA1
  /DestOutputProfile \the\pdflastobj\space 0 R
  /OutputConditionIdentifier (sRGB IEC61966-2.1)
  /Info(sRGB IEC61966-2.1)
 >> ]
}

\providecommand{\xmpOrg}{Tomas Bata University in Zlín, Czech Republic}
\providecommand{\xmpProducer}{}
\providecommand{\xmpDoi}{}
\providecommand{\xmpJournalnumber}{}
\providecommand{\xmpVolume}{}
\providecommand{\xmpIssue}{}
\providecommand{\xmpCoverDisplayDate}{}
\providecommand{\xmpCoverDate}{}
\providecommand{\xmpJournaltitle}{}
\providecommand{\xmpFirstpage}{}
\providecommand{\xmpLastpage}{}
\providecommand{\xmpAuthoritativeDomain}{}
\providecommand{\xmpCreatorTool}{}


% ============================================================================ %
% NASTAVENÍ TISKOVÉHO ZRCADLA

\newcommand{\valueTextHeight}{242mm}	% výška tiskového zrcadla
\newcommand{\valueTextWidth}{155mm}	% šířka tiskového zrcadla
\newcommand{\valueVOffset}{-1.61cm}	% vertikální posunutí tiskového zrcadla
\newcommand{\valueSideMargin}{0.96cm}	% levý okraj
\newcommand{\valueHeadHeight}{0.6cm}	% záhlaví
\newcommand{\valueHeadSep}{1cm}	% záhlaví

\textheight=\valueTextHeight
\textwidth=\valueTextWidth
\voffset=\valueVOffset
%\voffset=-1in
%\topmargin=-2.9cm

\oddsidemargin=\valueSideMargin
\evensidemargin=\valueSideMargin

\headheight=\valueHeadHeight
\headsep=\valueHeadSep

% nastavení zápatí
\footskip=1ex
\cfoot{}
% "vypnout" poznámky na okrajích
\marginparpush=0mm
\marginparwidth=0mm
\marginparsep=0mm

\pagestyle{fancy}

% Nastavení obalujících okrajů okolo popisků figur a tabulek
\captionsetup[figure]{aboveskip=10pt}
\captionsetup[figure]{belowskip=0pt}
\captionsetup[table]{aboveskip=0pt}
\captionsetup[table]{belowskip=-5pt}


% ============================================================================ %
% NASTAVENÍ PÍSMA, ODSTAVCE, ROVNIC, POZNÁMEK

\parindent=0em				% velikost odstavcové zarážky na nulu
\def\thefootnote{\arabic{footnote})}	% poznámka pod čarou se závorkou
\onehalfspacing % nastavím řádkování tímto způsobem nebo \renewcommand{\baselinestretch}{1.5} ??
\setlength{\parskip}{3pt}		% vertikální mezera mezi nadpisy
%\def\label#1{{\sf ! #1 ! }}		% možnost zobrazení všech \label{}


% ============================================================================ %
% NASTAVENÍ ČÍTAČŮ

\setcounter{tocdepth}{3} % do obsahu se ukládají pouze první dvě úrovně kapitol


% ============================================================================ %
% PDF/A STANDARD

% http://www.mathstat.dal.ca/~selinger/pdfa/
% https://blog.zhaw.ch/icclab/creating-pdfa-documents-for-long-term-archiving/
% http://support.river-valley.com/wiki/index.php?title=Generating_PDF/A_compliant_PDFs_from_pdftex

% Prerequisites: pdflatex, hyperref, xmpincl
% pdfTeX at least in version 1.40.15 (in Linux add repository ppa:jonathonf/texlive, update and upgrade texlive-full)
%
% Validator: https://www.pdf-online.com/osa/validate.aspx

\newcommand{\aplikujpdfa}{
	\ifczech
		\providecommand{\xmpTitle}{\nazevcz}
		\providecommand{\xmpAuthor}{\autor}
		\providecommand{\xmpKeywords}{\klicovaslovacz}
		\hypersetup{
			pdftitle={\nazevcz},
			pdfauthor={\autor},
			pdfsubject={\abstraktcz},
			pdfkeywords={\klicovaslovacz},
			pdflang={en}
		}
	\else \ifenglish
		\providecommand{\xmpTitle}{\nazeven}
		\providecommand{\xmpAuthor}{\autor}
		\providecommand{\xmpKeywords}{\klicovaslovaen}
		\hypersetup{
			pdftitle={\nazeven},
			pdfauthor={\autor},
			pdfsubject={\abstrakten},
			pdfkeywords={\klicovaslovaen},
			pdflang={en}
		}
	\fi \fi
	
	\makeatletter
	\includexmp{tex/pdfa-1b}
	\makeatother
}


% ============================================================================ %
% UŽIVATELSKÉ STYLY

% Styl nn = nečíslovaný nadpis (je vysázený v obsahu)
\def\nn#1{\clearpage\section*{\MakeTextUppercase{#1}}\addcontentsline{toc}{section}{#1}}

% Styl nm = nečíslovaný nadpis (není vysázený v obsahu)
\def\nm#1{\clearpage\section*{\MakeTextUppercase{#1}}}

% Styl ns = nečíslovaný nadpis na stejné stránce (není vysázený v obsahu)
\def\nns#1{\section*{\MakeTextUppercase{#1}}}

% Styl n{ur}{nadp} pro nadpisy, kde ur je číslo úrovně a nadp je text nadpisu
\def\n#1#2{
	\ifthenelse{#1=1}{\clearpage\section{#2}}{
		\ifthenelse{#1=2}{\subsection{#2}}{
			\ifthenelse{#1=3}{\subsubsection{#2}}{\paragraph{\itshape\bfseries{#2}}
}}}}

% Styl pro obrázky
% \obr{popisek}{label}{rozměr (0.0 - 1.0)}{soubor}
\def\obr#1#2#3#4{
	\begin{figure}[h]
		\centering
		\includegraphics[width=#3\linewidth]{#4}
		%\captionwidth{#3\linewidth}
		%\changecaptionwidth
		\captionsetup{width=#3\linewidth}
		\caption{#1}
		\label{#2}
	\end{figure}
}

% Styl pro tabulky
% \tab{popisek}{label}{rozměr (0.0 - 1.0)}{definice sloupců}{obsah} 
\def\tab#1#2#3#4#5{
	\begin{table}[h]
		%\captionwidth{#3\linewidth}
		%\changecaptionwidth
		\captionsetup{width=#3\linewidth}
		\caption{#1}
		\label{#2}
		\begin{center}
			\centering
			\begin{tabular}{#4}
				#5
			\end{tabular}
		\end{center}
	\end{table}
}

% Styl pro tabulky v příloze
% \tabpri{popisek}{definice sloupců}{data tabulky}
\def\tabpri#1#2#3{
	\begin{table}[h]
	\begin{center}
	#1
	\end{center}
	\begin{center}
	\begin{tabular}{#2}
	#3
	\end{tabular}
	\end{center}
	\end{table}
}
	
% Styl pro tabulky z MS Excelu exportované do EPS
% \extab{popisek}{rozměr (0.0 - 1.0)}{soubor}
\def\extab#1#2#3{
	\begin{table}
	%\captionwidth{#2\linewidth}
	%\changecaptionwidth
	\captionsetup{width=#2\linewidth}
	\caption{#1}
	\begin{center}
	\includegraphics[width=#2\linewidth]{#3}
	\end{center}
	\end{table}
}

% Styl pro rovnice
% \rov[klíčové slovo]{rovnice}
\newcommand{\rov}[2][chybejici rovnice]{
	\begin{equation}
	#2
	\label{#1}
	\end{equation}
}
	
% Příkaz pro vysázení seznamu obrázků
\def\seznamobr{
	\clearpage
	\ifczech
		\addcontentsline{toc}{section}{Seznam obrázků}
	\else \ifenglish
		\addcontentsline{toc}{section}{List of Figures}
	\fi \fi
	\listoffigures
	\clearpage
}

% Příkaz pro vysázení seznamu tabulek
\def\seznamtab{
	\clearpage
	\ifczech
		\addcontentsline{toc}{section}{Seznam tabulek}
	\else \ifenglish
		\addcontentsline{toc}{section}{List of Tables}
	\fi \fi
	\listoftables
	\clearpage
}

\newcommand{\OdsazovaniOdstavcuStart}[0]{
	\ifenglish
		\setlength{\parskip}{5mm} % English indentation of paragraphs
	\else \ifczech
		\setlength{\parindent}{5mm} % Czech indentation of paragraphs
	\fi \fi
}

\newcommand{\OdsazovaniOdstavcuStop}[0]{
	\ifenglish
		\setlength{\parskip}{0mm} % English indentation of paragraphs
	\else \ifczech
		\setlength{\parindent}{0mm} % Czech indentation of paragraphs
	\fi \fi
}

% Příkaz pro vysázení seznamu literatury
\newcommand{\seznamlit}[1]{
	\clearpage
	\ifczech
		\addcontentsline{toc}{section}{Seznam použité literatury}
	\else \ifenglish
		\addcontentsline{toc}{section}{References}
	\fi \fi
	\begin{thebibliography}{99}
	#1
	\end{thebibliography}
}

\newcommand{\seznamlitbib}{
	\bibliographystyle{tex/csplainnat} % Respects the norm of ČSN ISO 690
	\newpage
	\clearpage
	%\cleardoublepage
	\addcontentsline{toc}{section}{\protect\numberline{}{\ifenglish References \else \ifczech Seznam použité literatury \fi \fi}}
	\bibliography{tex/literatura}
}

% Příkaz pro přípravu seznamu použitých zkratek a symbolů
\newcommand{\seznamzkr}{
	\ifczech
		\nn{Seznam použitých symbolů a zkratek}
	\else \ifenglish
		\nn{List of Abbreviations}
	\fi \fi
}

% Příkaz \cast jako alternativa k \part
\def\cast#1{
	\part{#1}
}

% Příkaz \obsah vysází obsah v daném místě
\def\obsah{
	\deaktivujZahlavi
	\clearpage
	\thispagestyle{empty}
	\tableofcontents
	\clearpage
	\pagestyle{fancy}
	\aktivujZahlavi
}

% Zkrácení stylu \textbf na \b
\def\b#1{
	\textbf{#1}
}

% \bi = tučná kurzíva
\newcommand{\bi}[1]{\textbf{\textit{#1}}}

% \it = kurzíva
\renewcommand{\it}[1]{\textit{#1}}

% Nastaveni nezobrazovani zahlavi dokumentu
\newcommand{\deaktivujZahlavi}{
	\lhead{}
	\rhead{}
	\renewcommand{\headrulewidth}{0pt}
}

\newcommand{\zadani}{
	\clearpage
	\thispagestyle{empty}
	\voffset=\valueVOffset\evensidemargin=\valueSideMargin\oddsidemargin=\valueSideMargin\headsep=\valueHeadSep\headheight=\valueHeadHeight\setlength{\parskip}{3pt}\textheight=\valueTextHeight\textwidth=\valueTextWidth
	*** Nascanované zadání, strana 1 ***
	
	\clearpage
	\thispagestyle{empty}
	*** Nascanované zadání, strana 2 ***
}

% Nastaveni zobrazovani zahlavi dokumentu
\newcommand{\aktivujZahlavi}{
	\renewcommand{\headrulewidth}{1pt}
	\rhead{\thepage}
	
	\ifczech
		\lhead{\b{UTB ve Zlíně, \ifthenelse{\equal{\fakulta}{FAI}}{Fakulta aplikované informatiky}{\ifthenelse{\equal{\fakulta}{FAME}}{Fakulta managementu a ekonomiky}{\ifthenelse{\equal{\fakulta}{FHS}}{Fakulta humanitních studií}{\ifthenelse{\equal{\fakulta}{FLKR}}{Fakulta logistiky a krizového řízení}{\ifthenelse{\equal{\fakulta}{FMK}}{Fakulta multimediálních komunikací}{\ifthenelse{\equal{\fakulta}{FT}}{Fakulta technologická}{\ifthenelse{\equal{\fakulta}{UNI}}{Univerzitní institut}{}}}}}}}}}
	\else \ifenglish
		\lhead{\b{TBU in Zlín, \ifthenelse{\equal{\fakulta}{FAI}}{Faculty of Applied Informatics}{\ifthenelse{\equal{\fakulta}{FAME}}{Faculty of Management and Economics}{\ifthenelse{\equal{\fakulta}{FHS}}{Faculty of Humanities}{\ifthenelse{\equal{\fakulta}{FLKR}}{Faculty of Logistics and Crisis Management}{\ifthenelse{\equal{\fakulta}{FMK}}{Faculty of Multimedia Communications}{\ifthenelse{\equal{\fakulta}{FT}}{Faculty of Technology}{\ifthenelse{\equal{\fakulta}{UNI}}{University Institute}{}}}}}}}}}
	\fi \fi
}

% Příkaz \logopracerok vloží na dané místo logo fakulty, typ práce a rok
\newcommand{\logopracerok}{
	\ifczech
		\iffai  \put(82.2,-223.3){\makebox(84,16.4){\includegraphics[width=90mm]{graphics/logo/fai_logo_cz.png}}} \fi
		\iffame \put(82.2,-223.3){\makebox(84,16.4){\includegraphics[width=90mm]{graphics/logo/fame_logo_cz.png}}} \fi
		\iffhs  \put(82.2,-223.3){\makebox(84,16.4){\includegraphics[width=90mm]{graphics/logo/fhs_logo_cz.png}}} \fi
		\ifflkr \put(82.2,-223.3){\makebox(84,16.4){\includegraphics[width=90mm]{graphics/logo/flkr_logo_cz.png}}} \fi
		\iffmk  \put(82.2,-223.3){\makebox(84,16.4){\includegraphics[width=90mm]{graphics/logo/fmk_logo_cz.png}}} \fi
		\ifft   \put(82.2,-223.3){\makebox(84,16.4){\includegraphics[width=90mm]{graphics/logo/ft_logo_cz.png}}} \fi
		\ifuni  \put(82.2,-223.3){\makebox(84,16.4){\includegraphics[width=90mm]{graphics/logo/uni_logo_cz.png}}} \fi
	\else \ifenglish
		\iffai  \put(82.2,-223.3){\makebox(84,16.4){\includegraphics[width=90mm]{graphics/logo/fai_logo_en.png}}} \fi
		\iffame \put(82.2,-223.3){\makebox(84,16.4){\includegraphics[width=90mm]{graphics/logo/fame_logo_en.png}}} \fi
		\iffhs  \put(82.2,-223.3){\makebox(84,16.4){\includegraphics[width=90mm]{graphics/logo/fhs_logo_en.png}}} \fi
		\ifflkr \put(82.2,-223.3){\makebox(84,16.4){\includegraphics[width=90mm]{graphics/logo/flkr_logo_en.png}}} \fi
		\iffmk  \put(82.2,-223.3){\makebox(84,16.4){\includegraphics[width=90mm]{graphics/logo/fmk_logo_en.png}}} \fi
		\ifft   \put(82.2,-223.3){\makebox(84,16.4){\includegraphics[width=90mm]{graphics/logo/ft_logo_en.png}}} \fi
		\ifuni  \put(82.2,-223.3){\makebox(84,16.4){\includegraphics[width=90mm]{graphics/logo/uni_logo_en.png}}} \fi
	\fi \fi
	\put(0,-205){\linethickness{1pt}\line(1,0){170}}
	\ifczech
		\ifbp \put(4,-215){\makebox(69.5,4.5)[l]{\noindent\fontsize{16}{1}\usefont{OT1}{phv}{m}{n}Bakalářská práce}} \fi
		\ifdp \put(4,-215){\makebox(69.5,4.5)[l]{\noindent\fontsize{16}{1}\usefont{OT1}{phv}{m}{n}Diplomová práce}} \fi
	\else \ifenglish
		\ifbp \put(4,-215){\makebox(69.5,4.5)[l]{\noindent\fontsize{16}{1}\usefont{OT1}{phv}{m}{n}Bachelor's thesis}} \fi
		\ifdp \put(4,-215){\makebox(69.5,4.5)[l]{\noindent\fontsize{16}{1}\usefont{OT1}{phv}{m}{n}Master's thesis}} \fi
	\fi \fi
	\put(4,-220){\makebox(69.5,4.5)[l]{\noindent\fontsize{16}{1}\usefont{OT1}{phv}{m}{n}\rok}}
	\put(0,-225){\linethickness{1pt}\line(1,0){170}}
	\put(75,-223.3){\linethickness{1pt}\line(0,1){16.4}}
}

% Úvodní stránka s logem fakulty
\newcommand{\titulnistrana}{
	\thispagestyle{empty}
	\voffset=-2.01cm\evensidemargin=0pt\oddsidemargin=0cm\parindent=0pt\headsep=0pt\headheight=0pt\parskip=0pt\textheight=272mm\textwidth=200mm
	\renewcommand{\baselinestretch}{0}
	
	\setlength{\unitlength}{1mm}
	\begin{picture}(-10,8)
		\ifczech
			% Nazev prace
			%\put(0,-100){\makebox(170,50){\fontsize{24}{1}\usefont{OT1}{phv}{b}{n}#1}}
			%		\put(0,-100){\makebox(170,50){\protect\parbox{0.8\textwidth}{\protect\centering\fontsize{24}{1}\usefont{OT1}{phv}{b}{n}#1}}}
			
			% Vyreseno odradkovani
			\put(0,-100){\makebox(170,50){\protect\parbox{0.8\textwidth}{\protect\centering\setstretch{2.0}\usefont{OT1}{phv}{b}{n}{\Huge\nazevcz}}}}
			
			% Jmeno autora
			\put(0,-135){\makebox(170,25){\fontsize{20}{1}\usefont{OT1}{phv}{m}{n}\autor}}
		\else \ifenglish
			% Nazev prace
			%\put(0,-100){\makebox(170,50){\fontsize{24}{1}\usefont{OT1}{phv}{b}{n}#1}}
			%\put(0,-95){\makebox(170,50){\protect\parbox{0.8\textwidth}{\protect\centering\fontsize{24}{1}\usefont{OT1}{phv}{b}{n}#1}}}
			\put(0,-88){\makebox(170,50){\protect\parbox{0.8\textwidth}{\protect\centering\setstretch{2.0}\usefont{OT1}{phv}{b}{n}{\Huge\nazeven}}}}
	
			%\put(0,-111){\makebox(170,50){\fontsize{20}{1}\usefont{OT1}{phv}{m}{n}#1}}
			%\put(0,-116){\makebox(170,50){\protect\parbox{0.8\textwidth}{\protect\centering\fontsize{20}{1}\usefont{OT1}{phv}{m}{n}#2}}}
			\put(0,-115){\makebox(170,50){\protect\parbox{0.8\textwidth}{\protect\centering\setstretch{1.5}\usefont{OT1}{phv}{m}{n}{\Large\nazevcz}}}}
			
			% Jmeno autora
			\put(0,-140){\makebox(170,25){\fontsize{20}{1}\usefont{OT1}{phv}{m}{n}\autor}}
		\fi \fi
		\logopracerok
	\end{picture}
}


% Strana s abstraktem a klíčovými slovy v češtině a angličtině
\newcommand{\abstraktaklicovaslova}{
	\clearpage
	\thispagestyle{empty}
	\nm{Abstrakt}
	\abstraktcz
	
	\vspace{1cm}
	Klíčová slova: \klicovaslovacz
	
	\vspace{3cm}
	
	\nns{Abstract}
	\abstrakten
	
	\vspace{1cm}
	Keywords: \klicovaslovaen
}


% ============================================================================ %
% NASTAVENÍ ZOBRAZENÍ PŘÍLOH -- SEZNAM, ČÍSLOVÁNÍ, VLASTNÍ STYL

\makeatletter % tímto příkazem dávám najevo, že budu editovat přímo příkazy ze šablony

% definice seznamu příloh - příkaz \listofappendices
\def\listofappendices{%
	\newpage
	\setcounter{section}{0}
	\ifczech
		\addcontentsline{toc}{section}{SEZNAM PŘÍLOH}
		\@restonecolfalse\if@twocolumn\@restonecoltrue\onecolumn\fi
		\section*{SEZNAM PŘÍLOH}
	\else \ifenglish
		\addcontentsline{toc}{section}{LIST OF APPENDICES}
		\@restonecolfalse\if@twocolumn\@restonecoltrue\onecolumn\fi
		\section*{LIST OF APPENDICES}
	\fi \fi
	\@mkboth{LIST OF APPENDICES}{LIST OF APPENDICES}
	\@starttoc{loa}\if@restonecol\twocolumn\fi
	\pagestyle{empty}
	\thispagestyle{fancy}
}

\def\ext@appendix{loa}
\def\tocname{loa}

% definice příkazu \priloha{nazev prilohy} pro vložení nové přílohy
\newcommand{\priloha}[1]{
	\clearpage
	\refstepcounter{section}
	%\voffset=-3cm  % vertikalni posun
	\addtocontents{loa}{\protect\makebox[1.5cm][l]{P \@Roman\c@section.} #1\newline}
	\ifczech
		{\bf PŘÍLOHA P \@Roman\c@section. \MakeTextUppercase{#1}}
	\else \ifenglish
		{\bf APPENDIX P \@Roman\c@section. \MakeTextUppercase{#1}}
	\fi \fi
	\par
}

% ============================================================================ %
% OBSAH: NASTAVENÍ VELKÝCH PÍSMEN PRO NÁZVY SEKCÍ A HLAVNÍCH NADPISŮ

\let\oldcontentsline\contentsline
\def\contentsline#1#2{%
  \expandafter\ifx\csname l@#1\endcsname\l@section
    \expandafter\@firstoftwo
  \else
    \expandafter\@secondoftwo
  \fi
  {%
    \oldcontentsline{#1}{\MakeTextUppercase{#2}}%
  }{%
    \oldcontentsline{#1}{#2}%
  }%
}

\def\@part[#1]#2{
	\ifnum \c@secnumdepth >\m@ne
		\refstepcounter{part}
		\addcontentsline{toc}{section}{\protect\makebox[0.85cm]{\thepart\hfill} #1}
	\else
		\addcontentsline{toc}{section}{#1}
	\fi
	{\parindent \z@ \raggedright
	\interlinepenalty \@M
	\clearpage
	\normalfont
    \ifnum \c@secnumdepth >\m@ne
    	\Large\bfseries
		\nobreak
	\fi
	\vspace*{9cm}
	\center\huge \bfseries\thepart. \MakeTextUppercase{#2}
	\markboth{}{}\par}
	\nobreak
	\clearpage
    \@afterheading
}


% ============================================================================ %
% NASTAVENÍ FORMÁTU ČÍSLOVÁNÍ OBRÁZKŮ A TABULEK

\def\thefigure{\arabic{figure}}      % číslování obrázků typu (y)
\def\thetable{\arabic{table}}        % číslování tabulek typu (y)
\captiondelim{. } % změníme dvoutečku za Obr/Tab za tečku

% Nastavení číslování obrázků, tabulek i rovnic do formátu <číslo kapitoly>.<pořadové číslo>
\counterwithin{figure}{section}
\counterwithin{table}{section}
\counterwithin{equation}{section}

% Odsazeni popisku v seznamu obrazku a tabulek
\patchcmd{\@caption}{\csname the#1\endcsname}{\csname fnum@#1\endcsname}{}{}
%{\renewcommand*\numberline[1]{Fig. \,#1\space}}
\renewcommand*\l@figure{\@dottedtocline{1}{0em}{5.0em}}
\renewcommand*\l@table{\@dottedtocline{1}{0em}{5.0em}}

% Vynulování čítačů
\@addtoreset{table}{section}
\@addtoreset{figure}{section}
\@addtoreset{footnote}{section}
	
\makeatother % a to je ukončení \makeatletter


% ============================================================================ %
% ÚPRAVA VZHLEDU OBSAHU, SEZNAMU OBRÁZKŮ A TABULEK

% nastavení vertikální mezery před stylem část, nadpis 1--3
\setlength{\cftbeforepartskip}{3pt}
\setlength{\cftbeforesecskip}{3pt}
\setlength{\cftbeforesubsecskip}{3pt}
\setlength{\cftbeforesubsubsecskip}{0cm}

% odsazení zleva pro styl část, nadpis 1--3
\setlength{\cftpartindent}{0cm}
\setlength{\cftsecindent}{0cm}
\setlength{\cftsubsecindent}{0cm}
\setlength{\cftsubsubsecindent}{0cm}

% nastavení fontu pro styl část, nadpis 1--3
\renewcommand{\cftpartfont}{\small\bfseries}
\renewcommand{\cftsecfont}{\small\bfseries}
\renewcommand{\cftsubsecfont}{\scshape}
\renewcommand{\cftsubsubsecfont}{}

% odsazení čísla a textu titulku pro styl část, nadpis 1--3
\cftsetindents{part}{0cm}{1cm}
\cftsetindents{sec}{0cm}{1cm}
\cftsetindents{subsec}{0.5cm}{1.25cm}
\cftsetindents{subsubsec}{1cm}{1.5cm}
\cftsetindents{fig}{0cm}{1.5cm}
\cftsetindents{tab}{0cm}{1.5cm}

% nastavení vodící čáry pro styl část, nadpis 1--3, obrázky a tabulky
\renewcommand{\cftdot}{\ensuremath{.}} % tímto příkazem lze změnit vodící tečky v obsahu na jiný znak
\renewcommand{\cftpartleader}{\cftdotfill{0.3}}
\renewcommand{\cftsecleader}{\cftdotfill{0.3}}
\renewcommand{\cftsubsecleader}{\cftdotfill{0.3}}
\renewcommand{\cftsubsubsecleader}{\cftdotfill{0.3}}
\renewcommand{\cftfigleader}{\cftdotfill{0.3}}
\renewcommand{\cfttableader}{\cftdotfill{0.3}}

% změna fontu pro text "Obsah", "Seznam obrázků" a "Seznam tabulek"
\renewcommand{\cfttoctitlefont}{\normalsize\bfseries\thispagestyle{empty}}
\renewcommand{\cftloftitlefont}{\normalsize\bfseries\thispagestyle{fancy}}
\renewcommand{\cftlottitlefont}{\normalsize\bfseries\thispagestyle{fancy}}
\renewcommand{\cftfigpresnum}{Obr. }
\renewcommand{\cftfigaftersnum}{.}
\renewcommand{\cfttabpresnum}{Tab. }
\renewcommand{\cfttabaftersnum}{.}


% ============================================================================ %
% NASTAVENÍ FONTU PRO NADPISY

\sectionfont{\normalsize}
\subsectionfont{\normalsize\bfseries}
\subsubsectionfont{\small\bfseries}
\paragraphfont{\small\bf}

% definice nového stylu \comment -- komentář k šabloně
\newcommand{\comment}[1]{\color{red}#1\color{black}}


% ============================================================================ %
% VSTUPY

% Nastaveni a kontrola fakulty
\newcommand{\nastavfakultu}[1]{
	\newcommand{\fakulta}{#1}
	\newif\iffai  \let\iffai\iffalse
	\newif\iffame \let\iffame\iffalse
	\newif\iffhs  \let\iffhs\iffalse
	\newif\ifflkr \let\ifflkr\iffalse
	\newif\iffmk  \let\iffmk\iffalse
	\newif\ifft   \let\ifft\iffalse
	\newif\ifuni  \let\ifuni\iffalse
	
	\ifthenelse{\equal{#1}{FAI}}{\let\iffai\iftrue}{}
	\ifthenelse{\equal{#1}{FAME}}{\let\iffame\iftrue}{}
	\ifthenelse{\equal{#1}{FHS}}{\let\iffhs\iftrue}{}
	\ifthenelse{\equal{#1}{FLKR}}{\let\ifflkr\iftrue}{}
	\ifthenelse{\equal{#1}{FMK}}{\let\iffmk\iftrue}{}
	\ifthenelse{\equal{#1}{FT}}{\let\ifft\iftrue}{}
	\ifthenelse{\equal{#1}{UNI}}{\let\ifuni\iftrue}{}
	
	\iffai \else \iffame \else \iffhs \else \ifflkr \else \iffmk \else \ifft \else \ifuni \else
		\errmessage{Chyba nastaveni fakulty}
	\fi \fi \fi \fi \fi \fi \fi
}

% Nastaveni a kontrola typu prace
\newcommand{\nastavtyp}[1]{
	\newcommand{\typ}{#1}
	
	\newif\ifbp \let\ifbp\iffalse
	\newif\ifdp \let\ifdp\iffalse
	
	\ifthenelse{\equal{#1}{BP}}{\let\ifbp\iftrue}{}
	\ifthenelse{\equal{#1}{DP}}{\let\ifdp\iftrue}{}
	
	\ifbp \else \ifdp \else
		\errmessage{Chyba nastaveni typu prace}
	\fi \fi
}

% Nastaveni roku
\newcommand{\nastavrok}[1]{
	\newcommand{\rok}{#1}
}

% Nastaveni jmena
\newcommand{\nastavautora}[1]{
	\newcommand{\autor}{#1}
}

% Nastaveni nazvu
\newcommand{\nastavnazevcz}[1]{
	\newcommand{\nazevcz}{#1}
}
\newcommand{\nastavnazeven}[1]{
	\newcommand{\nazeven}{#1}
}

% Nastaveni abstraktu
\newcommand{\nastavabstraktcz}[1]{
	\newcommand{\abstraktcz}{#1}
}
\newcommand{\nastavabstrakten}[1]{
	\newcommand{\abstrakten}{#1}
}

% Nastaveni klicovych slov
\newcommand{\nastavklicovaslovacz}[1]{
	\newcommand{\klicovaslovacz}{#1}
}
\newcommand{\nastavklicovaslovaen}[1]{
	\newcommand{\klicovaslovaen}{#1}
}

% Nastaveni a kontrola jazyka
\newcommand{\nastavjazyk}[1]{
	\newcommand{\jazyk}{#1}
	
	\newif\ifczech   \let\ifczech\iffalse
	\newif\ifenglish \let\ifenglish\iffalse
	
	\ifthenelse{\equal{#1}{CZ}}{\let\ifczech\iftrue}{}
	\ifthenelse{\equal{#1}{EN}}{\let\ifenglish\iftrue}{}
	
	\ifczech \else \ifenglish \else
		\errmessage{Chyba nastaveni jazyka}
	\fi \fi
	
	\ifczech
		\usepackage[czech]{babel}
		% Vlastni definice nazvu
		\addto\captionsczech{\renewcommand{\contentsname}{\MakeTextUppercase{Obsah}}}
		\addto\captionsczech{\renewcommand{\refname}{\MakeTextUppercase{Seznam použité literatury}}}
		\addto\captionsczech{\renewcommand{\listfigurename}{\MakeTextUppercase{Seznam obrázků}}}
		\addto\captionsczech{\renewcommand{\listtablename}{\MakeTextUppercase{Seznam tabulek}}}
		\addto\captionsczech{\renewcommand{\figurename}{Obr.}}
		\addto\captionsczech{\renewcommand{\tablename}{Tab.}}
	\else \ifenglish
		\usepackage[english]{babel}	
		% Vlastni definice nazvu
		\addto\captionsenglish{\renewcommand{\contentsname}{\MakeTextUppercase{Table of Contents}}}
		\addto\captionsenglish{\renewcommand{\refname}{\MakeTextUppercase{References}}}
		\addto\captionsenglish{\renewcommand{\listfigurename}{\MakeTextUppercase{List of Figures}}}
		\addto\captionsenglish{\renewcommand{\listtablename}{\MakeTextUppercase{List of Tables}}}
		\addto\captionsenglish{\renewcommand{\figurename}{Fig.}}
		\addto\captionsenglish{\renewcommand{\tablename}{Tab.}}
	\fi \fi
}


% Nastaveni vertikalniho odsazeni nad rovnicemi/soustavami rovnic (prvni parametr),
% a pod (druhy parametr)
\newcommand{\nastavmezerukolemrovnic}[2]{
	\let\oldequation=\equation
	\let\endoldequation=\endequation
	\renewenvironment{equation}{\vspace{#1}\begin{oldequation}}{\end{oldequation}\vspace{#2}}
	
	\let\oldeqnarray=\eqnarray
	\let\endoldeqnarray=\endeqnarray
	\renewenvironment{eqnarray}{\vspace{#1}\begin{oldeqnarray}}{\end{oldeqnarray}\vspace{#2}}
}

% Nastaveni vertikalniho odsazeni nad tabulkami (prvni parametr),
% a pod (druhy parametr)
\newcommand{\nastavmezerukolemtabulek}[2]{
	\let\oldtable=\table
	\let\endoldtable=\endtable
	\renewenvironment{table}{\vspace{#1}\begin{oldtable}}{\end{oldtable}\vspace{#2}}
}

% Nastaveni vertikalniho odsazeni nad obrazky (prvni parametr),
% a pod (druhy parametr)
\newcommand{\nastavmezerukolemobrazku}[2]{
	\let\oldfigure=\figure
	\let\endoldfigure=\endfigure
	\renewenvironment{figure}{\vspace{#1}\begin{oldfigure}}{\end{oldfigure}\vspace{#2}}
}


% ============================================================================ %
% STRANA S PROHLASENIM

\newcommand{\prohlaseni}{{
	\clearpage
	\thispagestyle{empty}
	\ifenglish \nm{THESIS AUTHOR STATEMENT} \fi
	\textbf{Prohlašuji, že}
	\begin{itemize}
		\setlength{\parskip}{0pt}
		\setlength{\itemsep}{0pt}
		\setstretch{1.05}
		\item{beru na vědomí, že odevzdáním \ifbp bakalářské \else \ifdp diplomové \fi \fi práce souhlasím se zveřejněním své práce podle zákona č. 111/1998 Sb. o vysokých školách a o změně a doplnění dalších zákonů (zákon o vysokých školách), ve znění pozdějších právních předpisů, bez ohledu na výsledek obhajoby;}
		\item{beru na vědomí, že \ifbp bakalářské \else \ifdp diplomové \fi \fi práce bude uložena v elektronické podobě v univerzitním informačním systému dostupná k prezenčnímu nahlédnutí, že jeden výtisk \ifbp bakalářské \else \ifdp diplomové \fi \fi práce bude uložen v příruční knihovně \iffai Fakulty aplikované informatiky. \else \iffame Fakulty managementu a ekonomiky. \else \iffhs Fakulty humanitních studií. \else \ifflkr Fakulty logistiky a krizového řízení. \else \iffmk Fakutly mutimediálních komunikací. \else \ifft Fakulty technologické. \else \ifuni Univerzitního institutu. \if \fi \fi \fi \fi \fi \fi \fi \fi Univerzity Tomáše Bati ve Zlíně a jeden výtisk bude uložen u vedoucího práce; }
		\item{byl/a jsem seznámen/a s tím, že na moji \ifbp bakalářskou \else \ifdp diplomovou \fi \fi práci se plně vztahuje zákon č. 121/2000 Sb. o právu autorském, o právech souvisejících s právem autorským a o změně některých zákonů (autorský zákon) ve znění pozdějších právních předpisů, zejm. § 35 odst. 3;}
		\item{beru na vědomí, že podle § 60 odst. 1 autorského zákona má UTB ve Zlíně právo na uzavření licenční smlouvy o užití školního díla v rozsahu § 12 odst. 4 autorského zákona;}
		\item{beru na vědomí, že podle § 60 odst. 2 a 3 autorského zákona mohu užít své dílo – \ifbp bakalářskou \else \ifdp diplomovou \fi \fi práci nebo poskytnout licenci k~jejímu využití jen připouští-li tak licenční smlouva uzavřená mezi mnou a Univerzitou Tomáše Bati ve Zlíně s~tím, že vyrovnání případného přiměřeného příspěvku na úhradu nákladů, které byly Univerzitou Tomáše Bati ve Zlíně na vytvoření díla vynaloženy (až do jejich skutečné výše) bude rovněž předmětem této licenční smlouvy;}
		\item{beru na vědomí, že pokud bylo k vypracování \ifbp bakalářské \else \ifdp diplomové \fi \fi práce využito softwaru poskytnutého Univerzitou Tomáše Bati ve Zlíně nebo jinými subjekty pouze ke~studijním a výzkumným účelům (tedy pouze k~nekomerčnímu využití), nelze výsledky \ifbp bakalářské \else \ifdp diplomové \fi \fi práce využít ke komerčním účelům;}
		\item{beru na vědomí, že pokud je výstupem \ifbp bakalářské \else \ifdp diplomové \fi \fi práce jakýkoliv softwarový produkt, považují se za součást práce rovněž i zdrojové kódy, popř. soubory, ze kterých se projekt skládá. Neodevzdání této součásti může být důvodem k~neobhájení práce.}
	\end{itemize}
	
	\bigskip
	
%	\clearpage
%	\thispagestyle{empty}
	
	\textbf{Prohlašuji,}
	
	\begin{itemize}
		\setlength{\parskip}{0pt}
		\setlength{\itemsep}{0pt}
		\setstretch{1.05}
		\item{že jsem na \ifbp bakalářské \else \ifdp diplomové \fi \fi práci pracoval samostatně a použitou literaturu jsem citoval. V případě publikace výsledků budu uveden jako spoluautor.}
		\item{že odevzdaná verze \ifbp bakalářské \else \ifdp diplomové \fi \fi práce a verze elektronická nahraná do IS/STAG jsou totožné.}
	\end{itemize}
	
	\bigskip
	
	Ve Zlíně \hspace{7.2cm}\dots\dots\dots\dots\dots\dots\dots\dots\dots\dots
	
	\hspace{10.4cm}podpis autora
}}

% ============================================================================ %


% Uživatelské definice -- upravte dle požadavků
\nastavfakultu{FAI}
	% FAI  -- pro Fakultu aplikované informatiky
	% FAME -- pro Fakultu managementu a ekonomiky
	% FHS  -- pro Fakultu humanitních studií
	% FLKR -- pro Fakultu logistiky a krizového řízení
	% FMK  -- pro Fakutlu mutimediálních komunikací
	% FT   -- pro Fakultu technologickou
	% UNI  -- pro Univerzitní institut
\nastavtyp{DP}
	% BP   -- bakalářská práce
	% DP   -- diplomová práce
\nastavrok{2019}
	% zadejte rok místo "xxxx"
\nastavjazyk{CZ}
	% CZ   -- práce bude v českém jazyce
	% EN   -- práce bude v anglickém jazyce

% Lze přidat vertikalni odsazeni nad (prvni parametr) a pod (druhy parametr)
% obrázky, tabulky i rovnice/soustavy rovnic
\nastavmezerukolemobrazku{0mm}{0mm}
\nastavmezerukolemtabulek{0mm}{0mm}
\nastavmezerukolemrovnic{0mm}{0mm}

\nastavautora{Bc. Vojtěch Trefný}
\nastavnazevcz{Bitlocker šifrování disku v Linuxovém prostředí}
\nastavnazeven{Název práce anglicky (max. 2 řádky)} % Jen u anglicky psané práce
\nastavabstraktcz{Text abstraktu česky}
\nastavabstrakten{Text of the abstract}
\nastavklicovaslovacz{Přehled klíčových slov}
\nastavklicovaslovaen{Some keywords}

% Následující příkaz nastaví standard PDF/A-1b
\aplikujpdfa


% ============================================================================ %
\begin{document}

\titulnistrana

\zadani

\prohlaseni

\abstraktaklicovaslova


% ============================================================================ %
\clearpage
\thispagestyle{empty}
Zde je místo pro případné poděkování, motto, úryvky knih, básní atp.


% ============================================================================ %
\obsah  % Obsah je generován automaticky


% ============================================================================ %
\OdsazovaniOdstavcuStart % Nastaví odsazování odstavců dle zvoleného jazyka

% ============================================================================ %
% Encoding: UTF-8 (žluťoučký kůň úpěl ďábelšké ódy)
% ============================================================================ %

% ============================================================================ %
\nn{Úvod}
První odstavec pod nadpisem se neodsazuje, ostatní ano (pouze první řádek, odsazení vertikální mezy odstavci je typycké pro anglickou sazbu; czech babel toto respektuje, netřeba do textu přidávat jakékoliv explicitní formátování, viz ukázka sazby tohoto textu s následujícím odstavcem).

Formátování druhého odstavce. Text text text text text text text text text text text text.


% ============================================================================ %
\cast{Teoretická část}

\n{1}{BitLocker}
text

\n{2}{Diskový formát}

\todo{TODO: jak lépe říct on-disk format}

\n{3}{Hlavička}

Stejně jako u většiny diskových formátů, je i u BitLockeru na začátku disku takzvaná hlavička, která obsahuje základní informace o použitém formátu a jeho vlastnostech a také k jeho rychlé identifikaci. BitLocker hlavička zabírá celkem 512 bajtů a je u ní patrná inspirace u souborového systému NTFS. V tabulce \ref{tab:bitlocker-header} jsou zobrazeny jednotlivé (známé\footnote{Struktura formátu BitLocker není společností Microsoft nikde veřejně zcela kompletně zdokumentována, význam jednotlivých položek tedy nemusí být vždy přesně znám.}) položky hlavičky BitLockeru a pro srovnání také stejné položky v hlavičce souborového systému NTFS.

Struktura NTFS hlavičky je převzata z \cite{Carrier2005}, struktura BitLocker hlavičky je pak částečně převzata z \cite{j2YgJeTq3IrWJRcd}, částečně z \cite{Ferguson2006} a částečně výsledkem vlastního zkoumání.

\begin{table}[h]
\catcode`\-=12
\captionsetup{width=0.65\linewidth}
\caption{Porování položek hlaviček BitLocker a NTFS}
\label{tab:bitlocker-header}
\begin{center}
\centering
\begin{tabular}{|c|c|c|c|c|}
  \hline
   offset & velikost & BitLocker & NTFS \\ \hline
   0 & 3 & \multicolumn{2}{c|}{boot kód} \\ \hline
   3 & 8 & \multicolumn{2}{c|}{OEM název (signatura)} \\ \hline
   11 & 2 & \multicolumn{2}{c|}{počet bajtů na sektor} \\ \hline
   13 & 1 & \multicolumn{2}{c|}{počet sektorů na cluster} \\ \hline
   14 & 2 & \multicolumn{2}{c|}{rezervované sektory} \\ \hline
   16 & 4 & \multicolumn{2}{c|}{nepoužito} \\ \hline
   21 & 1 & \multicolumn{2}{c|}{popisek média} \\ \hline
   22 & 18 & \multicolumn{2}{c|}{nepoužito} \\ \hline
   40 & 8 & \multicolumn{2}{c|}{počet sektorů} \\ \hline
   48 & 8 & \multicolumn{2}{c|}{adresa prvního clusteru MFT} \\ \hline
   56 & 8 & \multicolumn{2}{c|}{kopie adresy prvního clusteru MFT} \\ \hline
   64 & 1 & \multicolumn{2}{c|}{velikost MFT entry} \\ \hline
   65 & 3 & \multicolumn{2}{c|}{nepoužito} \\ \hline
   68 & 1 & \multicolumn{2}{c|}{velikost indexu} \\ \hline
   69 & 3 & \multicolumn{2}{c|}{nepoužito} \\ \hline
   72 & 8 & \multicolumn{2}{c|}{NTFS serial number} \\ \hline
   80 & 4 & \multicolumn{2}{c|}{nepoužito} \\ \hline
   84 & 76 & \multicolumn{2}{c|}{boot kód} \\ \hline
   160 & 16 & BitLocker GUID & \multirow{4}{*}{boot kód} \\ \cline{1-3}
   176 & 8 & offset první kopie FVE metadat & \\ \cline{1-3}
   184 & 8 & offset druhé kopie FVE metadat & \\ \cline{1-3}
   192 & 8 & offset třetí kopie FVE metadat & \\ \hline
   200 & 310 & \multicolumn{2}{c|}{boot kód}  \\ \hline
   510 & 2 & \multicolumn{2}{c|}{signatura (\texttt{0xaa55})} \\ \hline
\end{tabular}
\end{center}
\end{table}

\begin{center}
\begin{lstlisting}[frame=none, escapechar=$, basicstyle=\ttfamily\small, columns=fullflexible, keepspaces=true, caption={BitLocker hlavička se zvýrazněnou signaturou, GUID a trojicí offsetů FVE metadat},label=lst:bitlocker-header]
00000000  eb 58 90 $\textcolor{blue}{2d 46 56 45 2d}$  46 53 2d 00 02 08 00 00  |.X.$\textcolor{blue}{-FVE-FS-}$.....|
00000010  00 00 00 00 00 f8 00 00  3f 00 ff 00 00 28 03 00  |........?....(..|
00000020  00 00 00 00 e0 1f 00 00  00 00 00 00 00 00 00 00  |................|
00000030  01 00 06 00 00 00 00 00  00 00 00 00 00 00 00 00  |................|
00000040  80 00 29 00 00 00 00 4e  4f 20 4e 41 4d 45 20 20  |..)....NO NAME  |
00000050  20 20 46 41 54 33 32 20  20 20 33 c9 8e d1 bc f4  |  FAT32   3.....|
00000060  7b 8e c1 8e d9 bd 00 7c  a0 fb 7d b4 7d 8b f0 ac  |{......|..}.}...|
00000070  98 40 74 0c 48 74 0e b4  0e bb 07 00 cd 10 eb ef  |.@t.Ht..........|
00000080  a0 fd 7d eb e6 cd 16 cd  19 00 00 00 00 00 00 00  |..}.............|
00000090  00 00 00 00 00 00 00 00  00 00 00 00 00 00 00 00  |................|
000000a0  3b d6 67 $\textcolor{green}{49 29 2e d8 4a}$  $\textcolor{green}{83 99 f6 a3 39 e3 d0 01}$  |;.gI)..J....9...|
000000b0  $\textcolor{red}{00 50 19 02 00 00 00 00}$  $\textcolor{orange}{00 d0 c1 02 00 00 00 00}$  |.P..............|
000000c0  $\textcolor{yellow}{00 a0 73 03 00 00 00 00}$  00 00 00 00 00 00 00 00  |..s.............|
000000d0  00 00 00 00 00 00 00 00  00 00 00 00 00 00 00 00  |................|
*
00000100  0d 0a 52 65 6d 6f 76 65  20 64 69 73 6b 73 20 6f  |..Remove disks o|
00000110  72 20 6f 74 68 65 72 20  6d 65 64 69 61 2e ff 0d  |r other media...|
00000120  0a 44 69 73 6b 20 65 72  72 6f 72 ff 0d 0a 50 72  |.Disk error...Pr|
00000130  65 73 73 20 61 6e 79 20  6b 65 79 20 74 6f 20 72  |ess any key to r|
00000140  65 73 74 61 72 74 0d 0a  00 00 00 00 00 00 00 00  |estart..........|
00000150  00 00 00 00 00 00 00 00  00 00 00 00 00 00 00 00  |................|
*
00000190  00 00 00 00 00 00 00 00  78 78 78 78 78 78 78 78  |........xxxxxxxx|
000001a0  78 78 78 78 78 78 78 78  78 78 78 78 78 78 78 78  |xxxxxxxxxxxxxxxx|
*
000001e0  78 78 78 78 78 78 78 78  ff ff ff ff ff ff ff ff  |xxxxxxxx........|
000001f0  ff ff ff ff ff ff ff ff  ff ff ff 00 1f 2c 55 aa  |.............,U.|
00000200
\end{lstlisting}
\end{center}

Z pohledu identifkace BitLocker zařízení je nejdůležitější částí hlavičky 8 bajtů na offsetu 3, které se u NTFS formátu nazývají \emph{OEM název} a které slouží pro rychlou identifikace zařízení. V linuxových systémech se podobné identifikátory obvykle nazývají \emph{signatura}. \todo{TODO: To by asi chtělo citaci.} Pro BitLocker formát je (u všech verzí) signatura v ASCII podobě \texttt{-FVE-FS-}.

Pro další práci s BitLockerem není většina položek hlavičky zajímavá. Výjimku tvoří GUID identifkátor uložený na offsetu 160 (16 bajtů dlouhý UTF-8 string) a trojice \emph{uint32} hodnot na offsetech 176, 184 a 192, které obsahují umístění (jako relativní offset od začátku zkoumaného zařízení) tří bloků FVE metadat. Všechny tyto čtyři hodnoty jsou v BitLocker hlavičce umístěny na offsetech, které jsou v NTFS součástí \emph{bootcode}.

Umístění všech výše zmíněných \uv{důležitých} částí BitLocker hlavičky je zobrazeno na výpisu\todo{TODO: Jak tomu sakra říkat} \ref{lst:bitlocker-header}.


\n{3}{FVE metadata}

Samotná výše popsaná hlavička formátu BitLocker neobsahuje o samotném BitLockeru téměř žádné informace. Slouží především pro rychlou identifikaci zařízení jako zařízení šifrovaného pomocí technologie BitLocker. Všechny informace potřebné pro práci s tímto zařízením, tedy především způsob uložení dat, jejich umístění, způsob jakým jsou šifrovány a hlavně klíč pro jejich (de)šifrování je uložený na třech různých místech\footnote{Na offsetech přibližně ve 33 \%, 44 \% a 55 \% u testovaných BitLocker zařízeních.} definovaných v hlavičce. Jedná se o tři identické kopie\footnote{Tři kopie jsou zvoleny pravděpodobně jako záloha pro případ náhodného poškození metadat. Vzhledem k tomu, že bez kompletní nepoškozené kopie těchto metadat není možné data na zařízení dešifrovat, je vícenásobná záloha na místě.} takzvaných \emph{FVE metadat}.

FVE metadata se skládají z celkem tří částí -- hlavičky FVE bloku (\emph{FVE metadata block header}), samotné FVE hlavičky (\emph{FVE metadata header}) a různého množství FVE záznamů (\emph{FVE metadata entry}, které obsahují samotné klíče a další důležité informace\cite{j2YgJeTq3IrWJRcd}\footnote{Toto dělení zavádí Joachim Metz v \cite{j2YgJeTq3IrWJRcd}. Teoreticky by se daly dvě první části metadat spojit, protože na disku se nachází vždy hned za sebou, ale rozdělení dává smysl, protože první část se týká popisu samotných metadat (signatura, verze, umístění všech tří bloků), zatímco druhá část už obsahuje samotná metadata (GUID, čas vytvoření, použitý šifrovací algoritmus).}.

Důležité položky v obou hlavičkách, jejich velikosti a offsety (vztažené vůči začátku dané hlavičky) jsou uvedeny v tabulce \ref{tab:fve-header}. Kompletní struktura obou hlaviček je součástí přílohy \todo{TODO: odkaz na přílohu}.

\tab{Zjednodušená struktura FVE metadat}{tab:fve-header}{0.65}{|l|l|l|}{
  \hline
   \multicolumn{3}{|c|}{\textbf{Hlavička FVE bloku}} \\ \hline
   offset & velikost & popis  \\ \hline
   0 & 8 & signatura (\texttt{-FVE-FS-}) \\ \hline
   10 & 2 & verze (1 nebo 2) \\ \hline
   32 & 8 & offset první kopie FVE metadat \\ \hline
   40 & 8 & offset druhé kopie FVE metadat \\ \hline
   48 & 8 & offset třetí kopie FVE metadat  \\ \hline
   \multicolumn{3}{}{} \\ \hline
   \multicolumn{3}{|c|}{\textbf{FVE hlavička}} \\ \hline
   0 & 4 & velikost metadat (včetně záznamů) \\ \hline
   16 & 16 & GUID \\ \hline
   36 & 4 & šifrovací algoritmus \\ \hline
   40 & 8 & datum a čas vytvoření \\ \hline
}

Mezi pro nás zajímavé položky v hlavičce patří její celková velikost (včetně velikosti samotné hlavičky a velikosti za ní následujícíh záznamů), šifrovací algoritmus použitý pro zašifrování dat uložených na disku (možné algoritmy jsou popsány v části \ref{sec:algorithms}) a v některých případech může být užitečný i čas vytvoření, který je uložen ve formátu \texttt{FILETIME}\footnote{FILETIME je ve skutečnosti struktura sestávající ze dvou 32bit integer hodnot, které dohromady udávají počet 100 nanosekundových intervalů, které k danému datu uplynuly od 1. ledna 1601.\cite{Zxwr6wjYZUQ6z8Yp}}.

\n{3}{FVE záznamy}

Za výše uvedenou hlavičkou se nechází blíže nespecifikované množství FVE záznamů. Ty slouží v podstatě jako key-value úložiště pro jakékoli další \uv{informace}, které jsou pro práci s BitLockerem potřebné. Tím, že není třeba předem určeno, kolik takových záznamů bude za hlavičkou uloženo, je možné přidávat nové položky při zachování zpětné kompatibility\footnote{Celková největší možná velikost FVE metadat je 64 KiB (alespoň tedy tolik je pro FVEm metadata vyhrazeno na vytvořených BitLocker zařízeních), teoreticky je tedy možné mít až 64 KiB - 112 B metadat.}.

Jelikož známe celkovou velikost FVE metadat (je uvedena v hlavičce, viz tabulka \ref{tab:fve-header}) a celková velikosti hlaviček FVE metadat je pevná (64 a 48 bajtů), pro přečtení všech záznamů stačí číst data ve smyččce, dokud nedojdeme na konec metadat, nebo dokud následující záznam nemá nulovou velikost.

Struktutra FVE je relativně jednoduchá a je popsaná v tabulce \ref{tab:fve-entry}. Důležitou součástí je velikost záznamu, protože podle svého typu může mít různou délku.

\tab{Struktura FVE záznamu}{tab:fve-entry}{0.65}{|l|l|l|}{
  \hline
   offset & velikost & popis  \\ \hline
   0 & 2 & velikost záznamu \\ \hline
   2 & 2 & typ záznamu \\ \hline
   4 & 2 & typ hodnoty záznamu \\ \hline
   6 & 2 & verze (1) \\ \hline
   8 &  & data  \\ \hline
}

Typ a hodnota označují, co je v daném záznamu uloženo. Známé typy a hodnoty jsou popsány v tabulce \ref{tab:fve-entry-types}. U typů se typicky jedná buď o klíč (FVEK, VMK) nebo obecnou property, hodnota pak dále specifikuje, jak je daný typ uložen (zašifrovaný klíč, unicode string).

Způsob uložení dat záleží na tom, jaká konkrétní data jsou v záznamu uložena. U \uv{jednoduchých} záznamů, jako je například popisek je v datech uložen textový řetězec uložený v kódování UTF-16, u \uv{složitějších} záznamů, jako jsou například klíče, mají data vlastní strukturu včetně dalších záznamů.

\begin{table}[h]
\catcode`\-=12
\captionsetup{width=0.65\linewidth}
\caption{Známé typy FVE záznamů}
\label{tab:fve-entry-types}
\begin{center}
\centering
\begin{tabular}{|l|l|c|l|l|}
  \cline{1-2} \cline{4-5}
   \multicolumn{2}{|c|}{\textbf{Typy}} &  & \multicolumn{2}{|c|}{\textbf{Hodnoty}} \\ \cline{1-2} \cline{4-5}
   typ & popis &  & typ & popis \\ \cline{1-2} \cline{4-5}
   0 & property & & 0 & smazáno \\ \cline{1-2} \cline{4-5}
   1 & VMK & & 1 & klíč \\ \cline{1-2} \cline{4-5}
   2 & FVEK & & 2 & string \\ \cline{1-2} \cline{4-5}
   7 & popisek & & 5 & AES-CCM šifrovaný klíč \\ \cline{1-2} \cline{4-5}
   15 & hlavička disku\footnotemark & & 6 & TPM klíč \\ \cline{1-2} \cline{4-5}
   \multicolumn{2}{c}{} & & 8 & VMK \\ \cline{4-5}
   \multicolumn{2}{c}{} & & 15 & offset a velikost \\ \cline{4-5}
   

\end{tabular}
\end{center}
\end{table}

\footnotetext{Umístění a velikost NTFS hlavičky otevřeného zařízení. Odpovídá hodnotě 15. Podrobnější informace o umístění NTFS hlavičky na šifrovaném zařízení jsou v části \ref{sec:data-map}.}

Příklad \uv{jednoduchého} záznamu je uveden na obrázku \ref{fig:fve-entry-desc}, kde vidíme záznam typu \emph{description}. Ten v podstatě obsahuje jméno počítače, na kterém bylo dané BitLocker zařízení vytvořeno a také datum vytvoření. Můžeme tedy vidět, že toto konkrétní BitLocker zařízení bylo vytvořeno na počítači \texttt{DESKTOP-NPM7RCA} a to 3. února 2019. Tato informace je uloženo jako standardní string v kódování UTF-16. Kromě tohoto stringu jsou pak na obrázku zvýrazněny i další údaje: velikost celého záznamu (64 bajtů), jeho typ (7 -- popisek) a hodnota (2 -- string) a verze (1).


\begin{figure}[h]
		\centering
		\captionsetup{width=0.65\linewidth}
		\caption{Příklad FVE záznamu typu \uv{description} (popisek)}
		\label{fig:fve-entry-desc}

\begin{lstlisting}[frame=none, escapechar=$, basicstyle=\ttfamily\small, columns=fullflexible, keepspaces=true]
02195070  $\textcolor{red}{40 00}$ $\textcolor{orange}{07 00}$ $\textcolor{yellow}{02 00}$ $\textcolor{green}{01 00}$  $\textcolor{blue}{44 00 45 00 53 00 4b 00}$ |@.......$\textcolor{blue}{D.E.S.K.}$|
02195080  $\textcolor{blue}{54 00 4f 00 50 00 2d 00}$  $\textcolor{blue}{4e 00 50 00 4d 00 37 00}$ |$\textcolor{blue}{T.O.P.-.N.P.M.7.}$|
02195090  $\textcolor{blue}{52 00 43 00 41 00 20 00}$  $\textcolor{blue}{47 00 3a 00 20 00 32 00}$ |$\textcolor{blue}{R.C.A. .G.:. .2.}$|
021950a0  $\textcolor{blue}{2f 00 33 00 2f 00 32 00}$  $\textcolor{blue}{30 00 31 00 39 00}$ 00 00 |$\textcolor{blue}{/.3./.2.0.1.9.}$..|
\end{lstlisting}

\end{figure}

U jednoduchého zařízení --- v našem případě USB flash disku --- se bude obvykle vyskytovat pouze pět záznamů a to již výše zmíněný popisek, dvojice záznamů typu VMK, jeden záznam typu FVEK (o obou více v části \ref{sec:keys} a jeden záznam obsahující informace o umístění hlavičky disku (o tomto záznamu více v části \ref{sec:data-map}).



\n{2}{Klíče}\label{sec:keys}

\n{3}{Volume Master Key}

\n{3}{Full Volume Encryption Key}

\n{2}{Šifrovaná data}

\n{3}{Použité šifrovací algoritmy}\label{sec:algorithms}

\n{3}{Způsob uložení data}\label{sec:data-map}

\n{3}{Postup při dešifrování}

\n{1}{Existující řešení pro práci s BitLockerem v Linuxu}

\n{2}{libbde}

\n{2}{Dislocker}


\n{1}{Další nadpis}
Tato sekce obsahuje ukázku vložení obrázku (Obr. \ref{fig:logo}).

% Obrázek lze vkládat pomocí následujícího zjednodušeného stylu, nebo klasickým LaTex způsobem
% Pozor! Obrázek nesmí obsahovat alfa kanál (průhlednost). Jde to proti standardu PDF/A.
\obr{Popisek obrázku}{fig:logo}{0.5}{graphics/logo/fai_logo_cz.png}


\n{2}{Podnadpis}
Tato sekce obsahuje ukázku vložení tabulky (Tab. \ref{tab:priklad}).

% Tabulku lze vkládat pomocí následujícího zjednodušeného stylu, nebo klasickým LaTex způsobem
\tab{Popisek tabulky}{tab:priklad}{0.65}{|l|c|c|c|c|c|r|}{
  \hline
   & 1 & 2 & 3 & 4 & 5 & Cena [Kč] \\ \hline
  \emph{F} & (jedna) & (dva) & (tři) & (čtyři) & (pět) & 300 \\ \hline
}

\n{3}{Podpodnadpis}

\n{3}{Podpodnadpis}
Citace knihy.


% ============================================================================ %
\cast{Projektová část}

\n{1}{Nadpis}

\n{2}{Podnadpis}


% ============================================================================ %
\nn{Závěr}
Text závěru


% ============================================================================ %
 % Hlavni text prace

\OdsazovaniOdstavcuStop


% ============================================================================ %
% Pro generování literatury lze alternativně použít i příkaz "\seznamlitbib", 
% který se postará o plnohodnotné vkládání referencí pomocí "bibliography". 
% V takovém případě se využívají bibliografické údaje uložené v souboru 
% tex-literatura.bib. Ty se automaticky upravuji dle zvolené citační normy 
% (v šabloně je nastavena korektní česká norma).
%\seznamlitbib
\seznamlit{
\bibitem{Ehrsam1978}
EHRSAM, William, Carl MEYER, John SMITH a Walter TUCHMAN. \textit{Message verification and transmission error detection by block chaining}. United States. US4074066A. Uděleno 1978-02-14. Zapsáno 1978-02-14.
\bibitem{Kohnoc2010}
KOHNO, Tadayoshi, Niels FERGUSON a Bruce SCHNEIER. \textit{Cryptography engineering: design principles and practical applications}. Indianapolis, IN: Wiley Pub., c2010. ISBN 978-0-470-47424-2.
\bibitem{WikimediaCBC}
Encryption using the Cipher Block Chaining (CBC) mode. In: \textit{Wikimedia Commons} [online]. [cit. 2019-04-22]. Dostupné z: https://commons.wikimedia.org/wiki/File:CBC\_encryption.svg
\bibitem{Regalado2013}
REGALADO, Daniel. CBC Byte Flipping Attack-101 Approach. In: \textit{Infosec Resources} [online]. Infosec Institute, 2013 [cit. 2019-04-22]. Dostupné z: https://resources.infosecinstitute.com/cbc-byte-flipping-attack-101-approach/
\bibitem{Ferguson2006}
FERGUSON, Niels. \textit{AES-CBC + Elephant diffuser: A Disk Encryption Algorithm for Windows Vista}. Microsoft, 2006.
\bibitem{Rosendorf2013}
ROSENDORF, Dan. The BitLocker schema with a view towards Windows 8. In: \textit{43. konference EurOpen.CZ} [online]. Plzeň, 2013, s.~91-101 [cit. 2019-03-04]. ISBN 978-80-86583-26-6. Dostupné z: https://europen.cz/Anot/43/eo-2-13.pdf
\bibitem{Kumar2008}
KUMAR, Nitin a Vipin KUMAR. Bitlocker and Windows Vista. In: \textit{NVLabs} [online]. 2008 [cit. 2019-04-20]. Dostupné z: http://www.nvlabs.in/uploads/projects/nvbit/nvbit\_bitlocker\_white\_paper.pdf
\bibitem{Dworkin2010}
DWORKIN, Morris. SP 800-38E. \textit{Recommendation for Block Cipher Modes of Operation: The XTS-AES Mode for Confidentiality on Storage Devices}. Gaithersburg, MD, United States: National Institute of Standards and Technology, 2010.
\bibitem{WikimediaXEX}
Schema of an XEX encryption. In: \textit{Wikimedia Commons} [online]. [cit. 2019-04-23]. Dostupné z: https://commons.wikimedia.org/wiki/File:XEX\_mode\_encryption.svg
\bibitem{IEEE2008}
IEEE STD 1619 -2007. \textit{IEEE Standard for Cryptographic Protection of Data on Block-Oriented Storage Devices}. New York, NY, United States: IEEE Computer Society, 2008.
\bibitem{ISO2009}
ISO/IEC 19772:2009. \textit{Information technology -- Security techniques -- Authenticated encryption}. Switzerland: International Organization for Standardization, 2009.
\bibitem{Dworkin2004}
DWORKIN, Morris. SP 800-38C. \textit{Recommendation for Block Cipher Modes of Operation: the CCM Mode for Authentication and Confidentiality}. Gaithersburg, MD, United States: National Institute of Standards and Technology, 2004. Dostupné také z: https://nvlpubs.nist.gov/nistpubs/Legacy/SP/nistspecialpublication800-38c.pdf
\bibitem{Housley2005}
HOUSLEY, R. IETF RFC 4309. \textit{Using Advanced Encryption Standard (AES) CCM Mode with IPsec Encapsulating Security Payload (ESP)}. Herndon, VA, United States: The Internet Society, 2005. Dostupné také z: https://www.ietf.org/rfc/rfc4309.txt
\bibitem{Kaliski2000}
KALISKI, B. IETF RFC 2898. \textit{PKCS \#5: Password-Based Cryptography Specification}. 2. Bedford (Massachuttes): RSA Laboratories, 2000. Dostupné také z: https://www.ietf.org/rfc/rfc2898.txt
\bibitem{Metz2011}
METZ, Joachim. BitLocker Drive Encryption (BDE) format specification. In: \textit{GitHub: Library and tools to access the BitLocker Drive Encryption (BDE) encrypted volumes} [online]. 2011 [cit. 2019-04-15]. Dostupné z: https://github.com/libyal/libbde/blob/master/documentation/BitLocker\\\%20Drive\%20Encryption\%20(BDE)\%20format.asciidoc
\bibitem{Agostini2019}
AGOSTINI, Elena a Massimo BERNASCHI. \textit{BitCracker: BitLocker meets GPUs} [online]. 2019. Dostupné z: https://arxiv.org/abs/1901.01337
\bibitem{VeraCrypt2019}
VeraCrypt Volume Format Specification. In: \textit{VeraCrypt Documentation} [online]. [cit. 2019-04-15]. Dostupné z: https://www.veracrypt.fr/en/VeraCrypt\%20Volume\%20Format\%20\\Specification.html
\bibitem{Carrier2005}
CARRIER, Brian. \textit{File system forensic analysis}. 1. London: Addison-Wesley, 2005. ISBN 978-0321268174.
\bibitem{Caseyc2010}
CASEY, Eoghan. \textit{Handbook of digital forensics and investigation}. Boston: Academic, c2010. ISBN 978-0123742674.
\bibitem{Zxwr6wjYZUQ6z8Yp}
Programming reference for Windows API: FILETIME structure. \textit{Microsoft Docs} [online]. [cit. 2019-04-07]. Dostupné z: https://docs.microsoft.com/en-us/windows/desktop/api/minwinbase/ns-minwinbase-filetime
\bibitem{Kornblum2009}
KORNBLUM, Jesse D. Implementing BitLocker Drive Encryption for forensic analysis. \textit{Digital Investigation}. 2009, \textbf{5}(3-4), 75-84. DOI: 10.1016/j.diin.2009.01.001. ISSN 17422876. Dostupné také z: https://linkinghub.elsevier.com/retrieve/pii/S1742287609000024
\bibitem{Lich2016}
LICH, Brian a Nick BROWER. Using Your PIN or Password: Changing your PIN or Password. In: \textit{Microsoft Docs} [online]. 2016 [cit. 2019-04-18]. Dostupné z: https://docs.microsoft.com/en-us/microsoft-desktop-optimization-pack/mbam-v2/using-your-pin-or-password
\bibitem{Zxwr6wjYZUQ6z8Yo}
Security WMI Providers: BitLocker Drive Encryption Provider. \textit{Microsoft Docs} [online]. [cit. 2019-04-09]. Dostupné z: https://docs.microsoft.com/en-us/windows/desktop/secprov/bitlocker-drive-encryption-provider
\bibitem{Hall2019}
HALL, Justin a Liza POGGEMEYER. BitLocker recovery guide. In: \textit{Microsoft Docs} [online]. 2019 [cit. 2019-04-15]. Dostupné z: https://docs.microsoft.com/en-us/windows/security/information-protection/bitlocker/bitlocker-recovery-guide-plan
\bibitem{hfTs55csrXKY7b4F}
Windows Vista support has ended. In: \textit{Windows Support} [online]. Microsoft, 2017 [cit. 2019-04-15]. Dostupné z: https://support.microsoft.com/en-us/help/22882/windows-vista-end-of-support
\bibitem{Sosnowski2016}
SOSNOWSKI, Rafal. Bitlocker: AES-3 (new encryption type). In: \textit{Microsoft TechNet} [online]. 2016 [cit. 2019-04-18]. Dostupné z: https://blogs.technet.microsoft.com/dubaisec/2016/03/04/bitlocker-aes-xts-new-encryption-type/
\bibitem{Metz2018}
METZ, Joachim. Library and tools to access the BitLocker Drive Encryption (BDE) encrypted volumes. In: \textit{GitHub} [online]. 2018 [cit. 2018-11-29]. Dostupné z: https://github.com/libyal/libbde
\bibitem{Coltel2017}
COLTEL, Romain a Hervé SCHAUER. Dislocker: FUSE driver to read/write Windows' BitLocker-ed volumes under Linux / Mac OSX. In: \textit{GitHub} [online]. 2017 [cit. 2019-04-15]. Dostupné z: https://github.com/Aorimn/dislocker
\bibitem{Metz2016}
METZ, Joachim. Building. In: \textit{Libbde Wiki} [online]. 2016 [cit. 2019-04-15]. Dostupné z: https://github.com/libyal/libbde/wiki/Building
\bibitem{Singh2014}
SINGH, Sumit. Develop your own filesystem with FUSE. In: \textit{IBM Developer: Linux Development} [online]. 2014 [cit. 2019-04-15]. Dostupné z: https://developer.ibm.com/articles/l-fuse/
\bibitem{Vangoor2017}
VANGOOR, Bharath Kumar Reddy a Vasily TARASOV. To FUSE or Not to FUSE: Performance of User-Space File Systems. In: \textit{FAST'17 Proceedings of the 15th Usenix Conference on File and Storage Technologies}. Santa Clara, CA, USA: USENIX Association Berkeley, 2017, s.~59-72. ISBN 978-1-931971-36-2.
\bibitem{GNU2019}
Frequently Asked Questions about the GNU Licenses: Is GPLv3 compatible with GPLv2?. In: \textit{GNU Project} [online]. [cit. 2019-04-15]. Dostupné z: https://www.gnu.org/licenses/gpl-faq.html.en\#v2v3Compatibility
\bibitem{RedHat2019}
Logical Volume Manager Administration: APPENDIX A. The Device Mapper. In: \textit{Red Hat Customer Portal} [online]. [cit. 2019-04-19]. Dostupné z: https://access.redhat.com/documentation/en-us/red\_hat\_enterprise\_linux/6/html/logical\_volume\_manager\_administration/\\device\_mapper
\bibitem{KernelZero}
\textit{Linux Kernel Documentation: dm-zero} [online]. In: . [cit. 2019-04-19]. Dostupné z: https://www.kernel.org/doc/Documentation/device-mapper/zero.txt
\bibitem{Broz2018}
BROŽ, Milan. Dm-crypt: Linux kernel device-mapper crypto target. In: \textit{Cryptsetup Wiki} [online]. 2018 [cit. 2019-04-19]. Dostupné z: https://gitlab.com/cryptsetup/cryptsetup/wikis/DMCrypt
\bibitem{Fruhwirth2005}
FRUHWIRTH, Clemens a Markus SCHUSTER. Secret Messages: Hard disk encryption with DM-Crypt, LUKS, and cryptsetup. \textit{Linux Magazine} [online]. 2005, \textbf{2005}(61) [cit. 2018-11-30]. ISSN 1752-9050. Dostupné z: https://nnc3.com/mags/LM10/Magazine/Archive/2005/61/065-071\_encrypt/article.html
\bibitem{Broz2014}
BROŽ, Milan. Cryptsetup 1.6.0 Release Notes. In: \textit{The Linux Kernel Archives} [online]. 2014 [cit. 2019-04-20]. Dostupné z: https://kernel.org/pub/linux/utils/cryptsetup/v1.6/v1.6.0-ReleaseNotes
\bibitem{GVfs2019}
GVfs. In: \textit{GNOME Wiki} [online]. 2019 [cit. 2019-04-18]. Dostupné z: https://wiki.gnome.org/Projects/gvfs
\bibitem{UDisks2018}
Udisks. In: \textit{FreeDesktop} [online]. 2018 [cit. 2019-04-18]. Dostupné z: https://www.freedesktop.org/wiki/Software/udisks/
\bibitem{Palmieri2005}
PALMIERI, John. Get on D-BUS. \textit{Red Hat Magazine} [online]. 2005, (3) [cit. 2019-04-18]. Dostupné z: http://www.redhat.com/magazine/003jan05/features/dbus/
\bibitem{Kenlon2018}
KENLON, Seth. An introduction to Udev: The Linux subsystem for managing device events. In: \textit{Opensource.com} [online]. 2018 [cit. 2019-04-18]. Dostupné z: https://opensource.com/article/18/11/udev
\bibitem{Zak2018}
ŽÁK, Karel. Util-linux 2.33 Release Notes. In: \textit{The Linux Kernel Archives} [online]. 2018 [cit. 2019-04-18]. Dostupné z: https://kernel.org/pub/linux/utils/util-linux/v2.33/v2.33-ReleaseNotes
\bibitem{Podzimek2015}
PODZIMEK, Vratislav. Libblockdev reaches the 1.0 milestone!. In: \textit{Storage APIs} [online]. 2015 [cit. 2019-04-18]. Dostupné z: https://storageapis.wordpress.com/2015/05/21/libblockdev-reaches-the-1-0-milestone/
\bibitem{Podzime2017}
PODZIMEK, Vratislav. UDisks to build on libblockdev!?. In: \textit{Storage APIs} [online]. 2017 [cit. 2019-04-18]. Dostupné z: https://storageapis.wordpress.com/2017/05/22/udisks-to-build-on-libblockdev/
\bibitem{Storaged2019}
\textit{UDisks Reference Manual}. 2019. Dostupné také z: http://storaged.org/doc/udisks2-api/latest/
}



% ============================================================================ %
% ============================================================================ %
% Encoding: UTF-8 (žluťoučký kůň úpěl ďábelšké ódy)
% ============================================================================ %

\seznamzkr

\begin{tabular}{ll}
  ABC & Význam zkratky \\
\end{tabular}

% ============================================================================ %
 % Seznam zkratek


% ============================================================================ %
\seznamobr  % Seznam je generován automaticky


% ============================================================================ %
\seznamtab  % Seznam je generován automaticky


% ============================================================================ %
% ============================================================================ %
% Encoding: UTF-8 (žluťoučký kůň úpěl ďábelšké ódy)
% ============================================================================ %

\listofappendices

\priloha{Název přílohy}
Obsah přílohy


% ============================================================================ %
 % Prilohy


% ============================================================================ %

\end{document}

% ============================================================================ %
