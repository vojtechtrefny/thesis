% ============================================================================ %
% Encoding: UTF-8 (žluťoučký kůň úpěl ďábelšké ódy)
% ============================================================================ %

\listofappendices

\priloha{Hlavička a FVE metadata BitLocker zařízení}\label{attachment:metadata}

Kompletní struktura FVE hlaviček. Uvedené hodnoty odpovídají testovacímu 100 MiB zařízení, které bylo vytvořeno ve Windows 10. Velikosti a offsety jsou uváděny v bajtech, pokud není uvedeno jinak. Popisy jednotlivých součástí hlaviček jsou převzaty z \cite{Metz2011}.

\section*{Hlavička FVE bloku}

\begin{table}[h]
\begin{center}
\centering
\begin{tabular}{|c|c|c|c|c|}
  \hline
   \textbf{offset} & \textbf{velikost} & \textbf{hodnota} & \textbf{popis} \\ \hline
   0 & 8 & -FVE-FS- & BitLocker signatura \\ \hline
   8 & 2 & 55 & velikost \\ \hline
   10 & 2 & 2 & verze \\ \hline
   12 & 2 & 4 & neznámé \\ \hline
   14 & 2 & 4 & neznámé \\ \hline
   16 & 8 & 104857600 & velikost zařízení v bajtech \\ \hline
   24 & 4 & 0 & neznámé \\ \hline
   28 & 4 & 16 & velikost hlavičky otevřeného zařízení \\ \hline
   32 & 8 & 35213312 & offset první kopie FVE metadat \\ \hline
   40 & 8 & 46256128 & offset druhé kopie FVE metadat \\ \hline
   48 & 8 & 57909248 & offset třetí kopie FVE metadat \\ \hline
   56 & 8 & 35278848 & offset hlavičky otevřeného zařízení \\ \hline
\end{tabular}
\end{center}
\end{table}

\section*{FVE hlavička}

\begin{table}[h]
\begin{center}
\centering
\begin{tabular}{|c|c|c|c|c|}
  \hline
   \textbf{offset} & \textbf{velikost} & \textbf{hodnota} & \textbf{popis} \\ \hline
   0 & 4 & 804 & velikost metadat \\ \hline
   4 & 4 & 1 & verze \\ \hline
   8 & 4 & 48 & velikost FVE hlavičky \\ \hline
   12 & 4 & 804 & velikost metadat -- kopie \\ \hline
   16 & 16 & 1f8bf933-...-a67625ac8f40 & GUID \\ \hline
   32 & 4 & 10 & následující hodnota nonce \\ \hline
   36 & 4 & 0x8004 & šifrovací metoda (AES-XTS) \\ \hline
   40 & 8 & 131936586222654059 & čas vytvoření (FILETIME) \\ \hline
\end{tabular}
\end{center}
\end{table}

\priloha{Manuálová stránka bitlockersetup}\label{attachment:manpage}


\begin{lstlisting}[frame=none, escapechar=$, columns=fullflexible, keepspaces=true, basicstyle=\ttfamily\small]
$\vspace{1em}$
$\pmb{NAME}$
$\vspace{2.5pt}$
       bitlockersetup - manage BitLocker encrypted devices
$\vspace{0.8em}$
$\pmb{SYNOPSIS}$
$\vspace{2.5pt}$
       bitlockersetup <options> <action> <action args>
$\vspace{0.8em}$
$\pmb{DESCRIPTION}$
$\vspace{2.5pt}$
       Bitlockersetup  is  a  tool  for  accessing  BitLocker devices in
       GNU/Linux using the Device Mapper crypto target.  Currently, only
       basic operations like open and close are being supported.

       Only password protected BitLocker devices that use AES-XTS encry-
       ption can be opened.  Older BitLocker  versions that use  AES-CBC
       and other protectors like TPM are not supported.
$\vspace{0.8em}$
$\pmb{COMMANDS}$
$\vspace{2.5pt}$
       open DEVICE [NAME]
              Open an existing BitLocker device using dm-crypt. The name
              argument is optional, if not  specified,  the  created  DM
              devices  will  be named as "bitlocker-UUID".  Password can
              also be provided on standard input if used  together  with
              the $\pmb{-q,} \pmb{-}\pmb{-quiet}$ option.

              The  newly created Device Mapper device /dev/mapper/<NAME>
              contains a standard NTFS filesystem that  can  be  mounted
              using ntfs-3g.
$\vspace{0.8em}$
       close NAME
$\vspace{2.5pt}$
              Closes an opened BitLocker device. This removes the exist-
              ing DM mapping NAME.
$\vspace{0.8em}$
       image DEVICE FILENAME
              Decrypts a BitLocker device and saves it as  an  image  to
              FILENAME.
$\vspace{0.8em}$
       dump DEVICE
              Prints  the header information about an existing BitLocker
              device.
$\vspace{1em}$
       uuid DEVICE
$\vspace{2.5pt}$
              Prints the UUID (GUID) of an existing BitLocker device.
$\vspace{0.8em}$
       isbitlocker DEVICE
$\vspace{2.5pt}$
              Checks if the  selected  device  is  a  BitLocker  device.
              Returns  true, if DEVICE is a BitLocker device, false oth-
              erwise.
$\vspace{0.8em}$
$\pmb{COMMON OPTIONS}$
$\vspace{2.5pt}$
       $\pmb{-h,} \pmb{-}\pmb{-help}$
$\vspace{2.5pt}$
              Show help text and default parameters.
$\vspace{0.8em}$
       $\pmb{-v,} \pmb{-}\pmb{-verbose}$
$\vspace{2.5pt}$
              Print more information on command execution.
$\vspace{0.8em}$
       $\pmb{-y,} \pmb{-}\pmb{-yes}$
              Do not prompt for confirmation  interactively  but  always
              assume the answer is yes.
$\vspace{0.8em}$
       $\pmb{-q,} \pmb{-}\pmb{-quiet}$
              Suppress output and log messages. Overrides --verbose.
$\vspace{0.8em}$
       $\pmb{-}\pmb{-version}$
$\vspace{2.5pt}$
              Show bitlockersetup version.
$\vspace{1em}$
bitlockersetup 0.1             April 2019              BITLOCKERSETUP(8)
\end{lstlisting}

\priloha{Struktura zdrojových kódů}\label{attachment:sources}


% ============================================================================ %
