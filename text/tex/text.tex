% ============================================================================ %
% Encoding: UTF-8 (žluťoučký kůň úpěl ďábelšké ódy)
% ============================================================================ %

% ============================================================================ %
\nn{Úvod}
První odstavec pod nadpisem se neodsazuje, ostatní ano (pouze první řádek, odsazení vertikální mezy odstavci je typycké pro anglickou sazbu; czech babel toto respektuje, netřeba do textu přidávat jakékoliv explicitní formátování, viz ukázka sazby tohoto textu s následujícím odstavcem).

Formátování druhého odstavce. Text text text text text text text text text text text text.


% ============================================================================ %
\cast{Teoretická část}

\n{1}{BitLocker}
text

\n{2}{Diskový formát}

\todo{TODO: jak lépe říct on-disk format}

\n{3}{Hlavička}

Stejně jako u většiny diskových formátů, je i u BitLockeru na začátku disku takzvaná hlavička, která obsahuje základní informace o použitém formátu a jeho vlastnostech a také k jeho rychlé identifikaci. BitLocker hlavička zabírá celkem 512 bajtů a je u ní patrná inspirace u souborového systému NTFS. V tabulce \ref{tab:bitlocker-header} jsou zobrazeny jednotlivé (známé\footnote{Struktura formátu BitLocker není společností Microsoft nikde veřejně zcela kompletně zdokumentována, význam jednotlivých položek tedy nemusí být vždy přesně znám.}) položky hlavičky BitLockeru a pro srovnání také stejné položky v hlavičce souborového systému NTFS.

Struktura NTFS hlavičky je převzata z \cite{Carrier2005}, struktura BitLocker hlavičky je pak částečně převzata z \cite{j2YgJeTq3IrWJRcd}, částečně z \cite{Ferguson2006} a částečně výsledkem vlastního zkoumání.

\begin{table}[h]
\catcode`\-=12
\captionsetup{width=1\linewidth}
\caption{Porování položek hlaviček BitLocker a NTFS}
\label{tab:bitlocker-header}
\begin{center}
\centering
\begin{tabular}{|c|c|c|c|c|}
  \hline
   offset & velikost & BitLocker & NTFS \\ \hline
   0 & 3 & \multicolumn{2}{c|}{boot kód} \\ \hline
   3 & 8 & \multicolumn{2}{c|}{OEM název (signatura)} \\ \hline
   11 & 2 & \multicolumn{2}{c|}{počet bajtů na sektor} \\ \hline
   13 & 1 & \multicolumn{2}{c|}{počet sektorů na cluster} \\ \hline
   14 & 2 & \multicolumn{2}{c|}{rezervované sektory} \\ \hline
   16 & 4 & \multicolumn{2}{c|}{nepoužito} \\ \hline
   21 & 1 & \multicolumn{2}{c|}{popisek média} \\ \hline
   22 & 18 & \multicolumn{2}{c|}{nepoužito} \\ \hline
   40 & 8 & \multicolumn{2}{c|}{počet sektorů} \\ \hline
   48 & 8 & \multicolumn{2}{c|}{adresa prvního clusteru MFT} \\ \hline
   56 & 8 & \multicolumn{2}{c|}{kopie adresy prvního clusteru MFT} \\ \hline
   64 & 1 & \multicolumn{2}{c|}{velikost MFT entry} \\ \hline
   65 & 3 & \multicolumn{2}{c|}{nepoužito} \\ \hline
   68 & 1 & \multicolumn{2}{c|}{velikost indexu} \\ \hline
   69 & 3 & \multicolumn{2}{c|}{nepoužito} \\ \hline
   72 & 8 & \multicolumn{2}{c|}{NTFS serial number} \\ \hline
   80 & 4 & \multicolumn{2}{c|}{nepoužito} \\ \hline
   84 & 76 & \multicolumn{2}{c|}{boot kód} \\ \hline
   160 & 16 & BitLocker GUID & \multirow{4}{*}{boot kód} \\ \cline{1-3}
   176 & 8 & offset první kopie FVE metadat & \\ \cline{1-3}
   184 & 8 & offset druhé kopie FVE metadat & \\ \cline{1-3}
   192 & 8 & offset třetí kopie FVE metadat & \\ \hline
   200 & 310 & \multicolumn{2}{c|}{boot kód}  \\ \hline
   510 & 2 & \multicolumn{2}{c|}{signatura (\texttt{0xaa55})} \\ \hline
\end{tabular}
\end{center}
\end{table}

\begin{center}
\begin{lstlisting}[frame=none, escapechar=$, basicstyle=\ttfamily\small, columns=fullflexible, keepspaces=true, caption={BitLocker hlavička se zvýrazněnou signaturou, GUID a trojicí offsetů FVE metadat},label=lst:bitlocker-header]
00000000  eb 58 90 $\textcolor{blue}{2d 46 56 45 2d}$  46 53 2d 00 02 08 00 00  |.X.$\textcolor{blue}{-FVE-FS-}$.....|
00000010  00 00 00 00 00 f8 00 00  3f 00 ff 00 00 28 03 00  |........?....(..|
00000020  00 00 00 00 e0 1f 00 00  00 00 00 00 00 00 00 00  |................|
00000030  01 00 06 00 00 00 00 00  00 00 00 00 00 00 00 00  |................|
00000040  80 00 29 00 00 00 00 4e  4f 20 4e 41 4d 45 20 20  |..)....NO NAME  |
00000050  20 20 46 41 54 33 32 20  20 20 33 c9 8e d1 bc f4  |  FAT32   3.....|
00000060  7b 8e c1 8e d9 bd 00 7c  a0 fb 7d b4 7d 8b f0 ac  |{......|..}.}...|
00000070  98 40 74 0c 48 74 0e b4  0e bb 07 00 cd 10 eb ef  |.@t.Ht..........|
00000080  a0 fd 7d eb e6 cd 16 cd  19 00 00 00 00 00 00 00  |..}.............|
00000090  00 00 00 00 00 00 00 00  00 00 00 00 00 00 00 00  |................|
000000a0  3b d6 67 $\textcolor{green}{49 29 2e d8 4a}$  $\textcolor{green}{83 99 f6 a3 39 e3 d0 01}$  |;.gI)..J....9...|
000000b0  $\textcolor{red}{00 50 19 02 00 00 00 00}$  $\textcolor{orange}{00 d0 c1 02 00 00 00 00}$  |.P..............|
000000c0  $\textcolor{yellow}{00 a0 73 03 00 00 00 00}$  00 00 00 00 00 00 00 00  |..s.............|
000000d0  00 00 00 00 00 00 00 00  00 00 00 00 00 00 00 00  |................|
*
00000100  0d 0a 52 65 6d 6f 76 65  20 64 69 73 6b 73 20 6f  |..Remove disks o|
00000110  72 20 6f 74 68 65 72 20  6d 65 64 69 61 2e ff 0d  |r other media...|
00000120  0a 44 69 73 6b 20 65 72  72 6f 72 ff 0d 0a 50 72  |.Disk error...Pr|
00000130  65 73 73 20 61 6e 79 20  6b 65 79 20 74 6f 20 72  |ess any key to r|
00000140  65 73 74 61 72 74 0d 0a  00 00 00 00 00 00 00 00  |estart..........|
00000150  00 00 00 00 00 00 00 00  00 00 00 00 00 00 00 00  |................|
*
00000190  00 00 00 00 00 00 00 00  78 78 78 78 78 78 78 78  |........xxxxxxxx|
000001a0  78 78 78 78 78 78 78 78  78 78 78 78 78 78 78 78  |xxxxxxxxxxxxxxxx|
*
000001e0  78 78 78 78 78 78 78 78  ff ff ff ff ff ff ff ff  |xxxxxxxx........|
000001f0  ff ff ff ff ff ff ff ff  ff ff ff 00 1f 2c 55 aa  |.............,U.|
00000200
\end{lstlisting}
\end{center}

Z pohledu identifkace BitLocker zařízení je nejdůležitější částí hlavičky 8 bajtů na offsetu 3, které se u NTFS formátu nazývají \emph{OEM název} a které slouží pro rychlou identifikace zařízení. V linuxových systémech se podobné identifikátory obvykle nazývají \emph{signatura}. \todo{TODO: To by asi chtělo citaci.} Pro BitLocker formát je (u všech verzí) signatura v ASCII podobě \texttt{-FVE-FS-}.

Pro další práci s BitLockerem není většina položek hlavičky zajímavá. Výjimku tvoří GUID identifkátor uložený na offsetu 160 (16 bajtů dlouhý UTF-8 string) a trojice \emph{uint32} hodnot na offsetech 176, 184 a 192, které obsahují umístění (jako relativní offset od začátku zkoumaného zařízení) tří bloků FVE metadat. Všechny tyto čtyři hodnoty jsou v BitLocker hlavičce umístěny na offsetech, které jsou v NTFS součástí \emph{bootcode}.

Umístění všech výše zmíněných \uv{důležitých} částí BitLocker hlavičky je zobrazeno na výpisu\todo{TODO: Jak tomu sakra říkat} \ref{lst:bitlocker-header}.

\n{3}{FVE metadata}

\n{2}{Klíče}

\n{3}{Volume Master Key}

\n{3}{Full Volume Encryption Key}

\n{2}{Šifrovaná data}

\n{3}{Použité šifrovací algoritmy}

\n{3}{Způsob uložení data}

\n{3}{Postup při dešifrování}

\n{1}{Existující řešení pro práci s BitLockerem v Linuxu}

\n{2}{libbde}

\n{2}{Dislocker}


\n{1}{Další nadpis}
Tato sekce obsahuje ukázku vložení obrázku (Obr. \ref{fig:logo}).

% Obrázek lze vkládat pomocí následujícího zjednodušeného stylu, nebo klasickým LaTex způsobem
% Pozor! Obrázek nesmí obsahovat alfa kanál (průhlednost). Jde to proti standardu PDF/A.
\obr{Popisek obrázku}{fig:logo}{0.5}{graphics/logo/fai_logo_cz.png}


\n{2}{Podnadpis}
Tato sekce obsahuje ukázku vložení tabulky (Tab. \ref{tab:priklad}).

% Tabulku lze vkládat pomocí následujícího zjednodušeného stylu, nebo klasickým LaTex způsobem
\tab{Popisek tabulky}{tab:priklad}{0.65}{|l|c|c|c|c|c|r|}{
  \hline
   & 1 & 2 & 3 & 4 & 5 & Cena [Kč] \\ \hline
  \emph{F} & (jedna) & (dva) & (tři) & (čtyři) & (pět) & 300 \\ \hline
}

\n{3}{Podpodnadpis}

\n{3}{Podpodnadpis}
Citace knihy.


% ============================================================================ %
\cast{Projektová část}

\n{1}{Nadpis}

\n{2}{Podnadpis}


% ============================================================================ %
\nn{Závěr}
Text závěru


% ============================================================================ %
